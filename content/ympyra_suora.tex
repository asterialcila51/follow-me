\section{Ympyrä ja suora}

\laatikko{
KIRJOITA TÄHÄN LUKUUN

\luettelo{
§ TEHTY - ympyrän ja suoran leikkauspisteiden ratkaiseminen 
§ ympyrän tangetin määrittäminen sekä kehällä olevan (kohtisuorassa sädettä vastaan TEHTY) että
sen ulkopuolisen pisteen kautta (kaksi tapaa: pisteen etäisyys suorasta -kaava (TEHTÄVÄKSI?) tai diskriminantti = 0 TEHTY)
§ TEHTY kahden ympyrän leikkauspisteiden ratkaiseminen yhtälöparilla
}

KIITOS!}

\begin{esimerkki}
Määritä suorien $x+2y-3=0$ ja ympyrän $(x-1)^2+(y+1)^2=4 $ leikkauspisteet.

\begin{esimratk}

Ympyrän ja suoran leikkauspisteet ovat ne pisteet $(x, y)$, jotka ovat sekä suoralla että ympyrällä, eli toteuttavat molempien yhtälöt, eli yhtälöryhmän
$$\left\{    
    \begin{array}{rcl}
        x+2y-3 &=&0 \\
        (x-1)^2+(y+1)^2 &=&4 \\
    \end{array}
    \right.$$
Vaikka yhtälö ei olekaan tutun lineaarinen, myös sitä voi lähestyä sijoitusmenetelmällä. Ratkaistaan ensimmäisestä yhtälöstä $x$ ja saadaan
\[
x = -2y+3
\]
Kun tämä sijoitetaan toiseen yhtälöön ja kerrotaan auki syntyy toisen asteen yhtälö $y$:n suhteen:
\begin{align*}
(-2y+3-1)^2+(y+1)^2=4 \\
(-2y+2)^2+(y+1)^2=4 \\
(-2y)^2-2\cdot 2y\cdot 2 +2^2+y^2+2\cdot y+1^2=4 \\
4y^2-8y+4+y^2+2y+1-4 = 0 \\
5y^2-6y+1 = 0
\end{align*}
Ratkaisukaavalla
\[
y = \frac{-(-6)\pm\sqrt{(-6)^2-4\cdot 5\cdot 1}}{2\cdot5}
\]
eli
\begin{align*}
y = \frac{6\pm\sqrt{16}}{10} \\
y = 1 \vee y = \frac{1}{5}
\end{align*}
Kun nämä $y$:n arvot sijoitetaan suoran yhtälöön saadaan vastaavast $x$:n arvot:
\begin{align*}
x = -2\cdot 1+3 \vee x = -2\cdot\frac{1}{5}+3 \\
x = 1 \vee x = \frac{13}{5}
\end{align*}
eli saatiin 2 ratkaisua: $(x, y) = (1, 1)$ ja $(x, y) = (\frac{13}{5}, \frac{1}{5})$. Tarkistamalla on hyvä vielä todeta, että pisteet todella ovat leikkauspisteitä.

\begin{kuva}
    kuvaaja.pohja(-3, 5, -4, 4, korkeus = 4, nimiX = "$x$", nimiY = "$y$", ruudukko = True)
    kuvaaja.piirraParametri("2*cos(t)+1","2*sin(t)-1", a = 0, b = 2*pi)
    kuvaaja.piirra("(-x+3)/2")
	
\end{kuva}
\begin{esimvast}
Leikkauspisteet ovat $(1, 1)$ ja $(\frac{13}{5}, \frac{1}{5})$.
\end{esimvast}

\end{esimratk}
\end{esimerkki}

Esimerkin avulla huomattiin, että suoralla ja ympyrällä voi olla kaksi leikkauspistettä. Suorasta ja ympyrästä riippuen päädytään toisen asteen yhtälöön, jolla on joko 0, 1 tai 2 ratkaisua. Jos leikkauspisteitä on tasan yksi, sanotaan, että suora sivuaa ympyrää tai suora on ympyrän \termi{tangentti}{tangentti}.

Kuva ympyrästä ja kolmesta suorasta; yksi tangentti, yksi leikkaa kahdessa pisteessä ja yksi ei leikkaa ympyrää.

\laatikko{Suoralla ja ympyrällä on nolla, yksi tai kaksi leikkauspisteitä.

Suora on ympyrän \emph{tangentti}, jos suoralla ja ympyrällä on tasan yksi yhteinen piste.
}

\begin{esimerkki}

Määritä $(2, 2)$-keskisen 5-säteisen ympyrän pisteen $(-5, 3)$ kautta kulkevat tangentit.

\begin{esimratk}
Suora on ympyrän tangentti, jos sillä ja ympyrällä on tasan yksi yhteinen piste. Jos pisteen $(-5, 3)$ suora ei ole $y$-akselin suuntainen, se voidaan esittää muodossa
\[
y-3 = k(x-(-5))
\]
eli
\[    
y = kx+5k+3.
\]
(Oletus voidaan tehdä, sillä selvästi nähdään ettei pisteen $(-5, 3)$ kautta kulkeva pystysuora suora ole kyseisen ympyrän tangentti.)

Kun lisäksi muistamme ympyrän yhtälön $(x-x_0)^2+(y-y_0)^2 = r^2$, suoran ja ympyrän leikkauspisteille saadaan yhtälöpari

\[
\left\{    
    \begin{array}{rcl}
        y &=& kx+5k+3 \\
        (x-2)^2+(y-2)^2 &=& 25 
\end{array}
    \right.
\]
    
Sijoitetaan ensimmäisen yhtälön lauseke $y$:lle toiseen yhtälöön ja saadaan toisen asteen yhtälö $x$:n suhteen
\begin{align*}
(x-2)^2+(kx+5k+3-2)^2&=25 \\
(x-2)^2+(kx+5k+1)^2&=25 \\
x^2-4x + 4+(kx)^2+kx\cdot 5k+kx \quad &\\
+5k\cdot kx+(5k)^2+5k+kx+5k+1& = 25\\
(k^2 +1)x^2+(10k^2+2k-4)x+25k^2+10k-20& = 0. \\
\end{align*}
Tällä yhtälöllä on $x$:n suhteen tasan yksi ratkaisu kun polynomin diskriminantti on nolla, eli
\begin{align*}
D &= b^2 -4ab \\
&= (10k^2+2k-4)^2-4\cdot(k^2 +1)\cdot(25k^2+10k-20) \\ 
&= 100k^4+20k^3-40k^2 + 20k^3 + 4k^2 -8k -40k^2 -8k +16 - 4(k^2 +1)(25k^2+10k-20)\\
&= 100k^4 + 40k^3 - 76k^2 - 16k + 16 - 4(25k^4+10k^3-20k^2 + 25k^2+10k-20) \\
&= 100k^4 + 40k^3 - 76k^2 - 16k + 16  - 100k^4 - 40k^3 - 20k^2 - 40k + 80 \\
& = -96k^2-56k+96 \\
&=-8(12k^2+7k-12) = 0.
\end{align*}
Toisen asteen yhtälön ratkaisukaavalla saadaan
\begin{align*}
k &= \frac{-7\pm \sqrt{7^2-4\cdot 12\cdot (-12)}}{2\cdot 12} \\
k &= \frac{-7\pm \sqrt{625}}{24} \\
k &= \frac{-7\pm 25 }{24}, \\
\end{align*}
joten
\begin{align*}
k = \frac{18}{24} =  \frac{3}{4} \; &\textrm{tai} \; k = -\frac{32}{24} = -\frac{4}{3}
\intertext{Kulmakertoimia vastaavat siis tangenttisuorat}
y = \frac{3}{4}x+5\cdot \frac{3}{4}+3 \; &\textrm{ja} \; y = -\frac{4}{3}x+5\cdot \Big(-\frac{4}{3}\Big)+3 \\
y = \frac{3}{4}x+\frac{27}{4} \; &\textrm{ja} \; y = -\frac{4}{3}x-\frac{11}{3}
\end{align*}
\end{esimratk}
\begin{esimvast}
Suorat $y = \frac{3}{4}x+\frac{27}{4}$ ja $y = -\frac{4}{3}x-\frac{11}{3}$
\end{esimvast}
\end{esimerkki}

Vastaavalla tavalla voidaan määrittää myös kahden ympyrän leikkauspisteet

\begin{esimerkki}
Määritä ympyröiden $(x-1)^2+(y+5)^2 = 13$ ja $(x+2)^2+(y+4)^2 = 17$ leikkauspisteet.

\begin{esimratk}
Ympyröiden leikkauspisteet toteuttavat yhtälöparin
\[
\left\{    
    \begin{array}{rcl}
        (x-1)^2+(y+5)^2 = 13 \\
        (x+2)^2+(y+4)^2 = 17 \\
    \end{array}
    \right.
\]
Kun binomien neliöt kerrotaan auki, sievennetään ja yhtälöt vähennetään toisistaan saadaan $x$:n ja $y$:n välille ensimmäisen asteen yhtälö
\[
\left\{    
    \begin{array}{rcl}
        x^2-2x+1+y^2+10y+25 = 13 \\
        x^2+4x+4+y^2+8y+16 = 17 \\
    \end{array}
    \right.
\]
\[
\left\{    
    \begin{array}{rcl}
        x^2-2x+y^2+10y= -13 \\
        x^2+4x+y^2+8y= -3 \\
    \end{array}
    \right.
\]
joten
\[
-6x+2y=-10
\]
Tämä suoran yhtälö vastaa oikeastaan ympyröiden leikkauspisteiden kautta kulkevaa suoraa. Tästä voidaan ratkaista $y$ $x$:n suhteen ja sijoittaa jompaan kumpaan alkuperäisistä yhtälöistä.
\begin{align*}
y = 3x-5 \\
(x-1)^2+((3x-5)+5)^2 = 13 \\
x^2-2x+1+(3x)^2 = 13 \\
10x^2-2x-12 = 0 \\
5x^2-x-6 = 0 \\
\end{align*}
Nyt leikkauspisteiden $x$-koordinaatit voidaan ratkaista toisen asteen yhtälön ratkaisukaavalla
\begin{align*}
x &= \frac{-(-1)\pm\sqrt{(-1)^2-4\cdot 5\cdot (-6)}}{2\cdot 5} \\
&= \frac{1\pm\sqrt{121}}{10} \\
&= \frac{1\pm 11}{10} \\
\end{align*}
eli
\[
x =  \frac{6}{5} \textrm{  tai  } x = -1
\]
Nyt suoran yhtälöstä voidaan ratkaista vastaavat $y$:n arvot.
\begin{align*}
y = 3\cdot \frac{6}{5}-5 &\textrm{  tai  } y = 3\cdot (-1)-5 \\
y = -\frac{7}{5}x &\textrm{  tai  } y = -8
\end{align*}
\end{esimratk}
\begin{esimvast}
Ympyröiden leikkauspisteet ovat $(\frac{6}{5}, -\frac{7}{5})$ ja $(-1,-8)$.
\end{esimvast}
\end{esimerkki}

Jälleen esimerkistä nähdään, että riippuen syntyvän toisen asteen yhtälön diskriminantista kahdella ympyrällä voi olla $0$, $1$ tai $2$ leikkauspistettä.


%%FIXME: kuvan voisi päivittää liittymään esimerkkiin?
\begin{kuva}
    kuvaaja.pohja(-7, 5, -11, 3, korkeus = 5, nimiX = "$x$", nimiY = "$y$", ruudukko = True)
    kuvaaja.piirraParametri("3.606*cos(t)+1","3.606*sin(t)-5", a = 0, b = 2*pi)
    kuvaaja.piirraParametri("4.123*cos(t)-2","4.123*sin(t)-4", a = 0, b = 2*pi)
	
\end{kuva}

Ympyrän tangentti saadaan yksikäsitteisesti myös kun tiedetään ympyrän kehän piste jossa tangentti sivuaa ympyrää:

\begin{esimerkki}
Mikä on ympyrää $(x-3)^2 + (y -1)^2 = 18$ pisteessä $(6,4)$ sivuavan tangentin yhtälö?
\begin{esimratk}
Koska tangentti kulkee pisteen $(6,4)$ kautta, jolloin sen yhtälö on 
\begin{align*}
y - 4 &= k_1(x-6) \\
y &= k_1x -6k_1 +4.
\end{align*}
Voisimme sijoittaa tämän edellisten esimerkkien tapaan ympyrän yhtälöön, mutta toinen (tässä tapauksessa helpompi) tapa on huomata tangentin olevan kohtisuorassa ympyrän sädettä eli keskipisteen $(3, 1)$ (jonka voimme lukea suoraan ympyrän yhtälöstä) ja pisteen $(6,4)$ välistä janaa vasten. Nämä kaksi pistettä määrittävät suoran, jonka kulmakerroin $k_2$ on
\begin{align*}
k_2 = \frac{y_2 - y_1}{x_2 - x_1} = \frac{4-1}{6-3} = \frac{3}{3} = 1.
\end{align*}
Koska suoran ja sen normaalin kulmakertoimien tulo on $k_1 k_2 = -1$, saadaan tangentin kulmakertoimeksi $k_1 = k_1 \cdot 1 = -1$, jonka voimme sijoittaa tangentin yhtälöön.
\end{esimratk}
\begin{esimvast}
Tangentin yhtälö on $y = -x +10$.
\end{esimvast}
\end{esimerkki}

%%%KUVA? %%%https://www.wolframalpha.com/input/?i=y%20%3D%20%2Dx%20%2B10%2C%20(x%2D3)%5E2%20%2B%20(y%20%2D1)%5E2%20%3D%2018

\begin{tehtavasivu}

\subsubsection*{Opi perusteet}

\begin{tehtava}
Määritä suoran ja ympyrän leikkauspisteet, jos suoran yhtälö, ja ympyrän keskipiste ja säde ovat
\alakohdat{
§ $3x-2y = 1$, $(-1,2)$, $4$
§ $x+1 = 0$, $(-6,-6)$, $10$
§ $x+7y-23 = 0$, $(7,-3)$, $5$
§ $5x+12y-13 = 0$, $(0,0)$, $5$
}
vastaavasti.
\begin{vastaus}
\alakohdat{
§ $(-1,-2)$ ja $(\frac{35}{13},\frac{46}{13})$
§ $(-1,-6+5\sqrt{3})$, $(-1,-6-5\sqrt{3})$
§ Käyrät eivät leikkaa
§ $(\frac{5}{13},\frac{12}{13})$
}
\end{vastaus}
\end{tehtava}

\begin{tehtava}
Määritä kaikki ympyröiden $ (x-3)^2+y^2= 10$, $(x+1)^2+(y-3)^2 = 18$ sekä $(x-6)^2+(y-2)^2 = 8$ leikkauspisteet (pareittain).
\begin{vastaus}
$(\frac{82-9\sqrt{79}}{50},-3\frac{4\sqrt{79}-17}{50})$, $(\frac{82+9\sqrt{79}}{50},3\frac{4\sqrt{79}+17}{50})$,
$(\frac{123-2\sqrt{295}}{26}, 3\frac{10+\sqrt{295}}{26})$
$(\frac{123+2\sqrt{295}}{26}, 3\frac{10-\sqrt{295}}{26})$
$(\frac{16}{5}, \frac{12}{5})$
\end{vastaus}
\end{tehtava}

\begin{tehtava}
Määritä parametrin $k$ arvot siten, että suora $y=x+k$ on ympyrän  $ x^2+y^2= 2$ tangentti.
\begin{vastaus}
$k = \pm sqrt{2} $,
\end{vastaus}
\end{tehtava}


\subsubsection*{Hallitse kokonaisuus}
\begin{tehtava}
	Määritä annetun ympyrän tangentit, jotka kulkevat annetun pisteen kautta.
	\alakohdat{
		§ Piste $(1, -1)$, ympyrä $(x-3)^2 + (y+1)^2 = 2$
		§ Piste $(-4, -2)$, ympyrä $(x+5)^2 + (y+6)^2 = 17$
		§ Piste $(3, 1)$, ympyrä $(x-4)^2 + y^2 = 1$
		§ Piste $(2, 4)$, ympyrä $x^2 + (y-3)^2 = 8$
	}
	\begin{vastaus}
		\alakohdat{
			§ $y=x-2$ ja $y=-x$
			§ $y= -1/4 x -3$
			§ $y=1$ ja $x=3$
			§ Ei ole.
		}
	\end{vastaus}
\end{tehtava}

\begin{tehtava}
Määritä yksikköympyrän $x^2+y^2= 1$ pisteeseen $(x_{0}, y_{0} )$ piirretyn tangentin normaalimuotoinen yhtälö. Entä jos ympyrä on $r$-säteinen?
\begin{vastaus}
$x_0x+y_0y=1 $. $r$-säteisellä ympyrällä $x_0x+y_0y=r^2$
\end{vastaus}
\end{tehtava}

\begin{tehtava}
Yksikköympyrälle $x^2+y^2=1$ piirretään tangentit pisteestä $(0, a)$. Millä $a$:n arvoilla tangentteja on 
\alakohdat{
§ kaksi
§ yksi
§ nolla?
}
Määritä myös tangenttien yhtälöt.
\begin{vastaus}
\alakohdat{
§ $a > 1$ tai $a < -1$
§ $a = \pm1$
§ $ -1 < a < 1 $ 
}
Tangenttien yhtälöt ovat $ y = \pm \sqrt{a^2-1}x+a$
\end{vastaus}
\end{tehtava}

\begin{tehtava}
\alakohdat{
§ Pisteen $P$ etäisyys $O$-keskisestä $r$-säteisestä ympyrästä on $d$ $(d > r) $. Kuinka pitkiä ovat ympyrälle pisteestä $P$ piirretyt tangentit?
§ Yksikköympyrälle ($x^2+y^2 = 1$) ja ympyrälle $(x+3)^2+(y-2)^2 = 2$ piirretään tangentit. Määritä kaikki pisteet $P = (x,y)$, joista piirretyt tangentit ovat yhtä pitkät. (Tätä pistejoukkoa kutsutaan yleensä ympyröiden \emph{radikaaliakseliksi}.)
}

\begin{vastaus}
\alakohdat{
§ $\sqrt{d^2-r^2}$
§ $3x-2y+6$
}
\end{vastaus}
\end{tehtava}

\begin{tehtava}
Määritä kaikkien pisteen $(1,0)$ kautta kulkevien ympyrän $x^2+y^2 = 4$ jänteiden keskipisteiden joukko.
	\begin{vastaus}
		Ympyrä $(x-\frac{1}{2})^2+y^2 = \frac{1}{4}$. Vinkki: Parametrisoi kaikki 			pisteen $(1,0)$ kautta kulkevat suorat, jatutki miten keskipisteiden $x$ ja 		$y$ -koordinaatit riippuvat parametrista.
	\end{vastaus}
\end{tehtava}

\subsubsection*{Sekalaisia tehtäviä}


TÄHÄN TEHTÄVIÄ SIJOITTAMISTA ODOTTAMAAN

\begin{tehtava}
Suoran ulkopuolisesta pisteestä $P$ piirrettyjen tangenttien ja ympyrän sivuamispisteet voidaan määrittää myös monella muulla tavalla:

Pisteen etäisyys suorasta -kaavalla: Jos suora on ympyrän tangentti, sen etäisyyden suorasta on oltava yhtä suuri kuin ympyrän säde.

Määrittämällä tangenttien pituudet: Koska tangentit ja ympyrän säde ovat kohtisuorassa, tangentien pituus voidaan määrittää ympyrän keskipisteen, pisteen $P$ ja sivuamispisteen muodostamasta suorakulmaisesta kolmiosta. Nyt sivuamispisteet voidaan määrittää kahden ympyrän leikkauspisteinä.

Klassiseen tyyliin Thaleen lausetta hyödyntäen: Voidaan osoittaa, että jos $M$ on pisteiden $P$ ja ympyrän keskipisteen $O$ keskipiste, $M$-keskinen ympyrä, joka kulkee pisteiden $P$ ja $O$ kautta leikkaa alkuperäistä ympyrää halutuissa sivuamispisteissä.

Parametrisoimalla kaikki tangentit: Kaikki ympyrän tangentit voidaan myös parametrisoida valitsemalla ympyrältä mielivaltainen piste ja määrittämällä sen kautta kulkeva tangentti (ks. teht ??). Sitten riittää tarkistaa, mitkä tangenteista kulkevat pisteen $P$ kautta.

Malliratkaisun variaatiolla: Parametrisoidaan kaikki $P$:n kautta kulkevat. Nyt saadaan malliratkaisun tavoin yhtälö leikkauspisteille. Sen sijaan, että nyt määritettäisiin, milloin syntyvän toisen asteen yhtälön diskriminantti on nolla, oletetaan, että se on nolla, jolloin leikkauspisteet saadaan määritettyä parametrin funktiona. Nyt riittää enää tarkistaa, milloin nämä pisteet todellakin ovat ympyrällä.

\alakohdat{
§ Jos $P = (0,0)$, $O = (6,4)$ ympyrän säde on $5$, määritä sivuamispisteet näillä viidellä tavalla.
§ Määritä vastaavien tangenttien yhtälöt.
}
\begin{vastaus}
\alakohdat{
§ $(\frac{81-30\sqrt{3}}{26},\frac{54+45\sqrt{3}}{26})$ ja $(\frac{81+30\sqrt{3}}{26},\frac{54-45\sqrt{3}}{26})$
§ $(27-10\sqrt{3})y-(18+15\sqrt{3})x = 0$ ja $(27+10\sqrt{3})y-(18-15\sqrt{3})x = 0$
}
\end{vastaus}
\end{tehtava}

\begin{tehtava}
Kaksi ympyrää leikkaa kohtisuorasti, jos ne leikkaavat, ja niiden leikkauspisteisiin piirretyt tangentit ovat kohtisuorassa. Leikkaavatko ympyrät kohtisuorasti
\alakohdat{
§ $(x+3)^2+(y+3)^2= 13$ ja $(x+2)^2+(y-1)^2 = 5 $
§ $(x-2)^2+y^2 = 16$ ja $(x+3)^2+y^2 = 9$
§ $(x+7)^2+(y-2)^2 = 23$ ja $(x+4)^2+(y+3)^2 = 1$?
}  
\begin{vastaus}
\alakohdat{
§ Eivät
§ Kyllä
§ Eivät
}
\end{vastaus}
\end{tehtava}

\begin{tehtava}
Matti ajaa autolla $(1,0)$-keskistä ympyrärataa, mutta kuin taikaiskusta renkaista lähtee pito ja kauhukseen Matti alkaa luisua irtoamispisteestä ympyräradan tangentin suuntaan. Matti saa pysäytettyä autonsa pisteeseen $(5,5)$. Kuinka pitkä ympyrärata on, jos se on yhtä pitkä kuin Matin luisuma matka?
\begin{vastaus}
$\frac{2\sqrt{41}\pi}{\sqrt{4\pi^2+1}}$
\end{vastaus}
\end{tehtava}

\begin{tehtava}
Todista analyyttisen geometrian keinoin tangenttien tärkeät perusominaisuudet
\alakohdat{
§ Ympyrälle piirretty tangentti ja tangentin sivuamispisteeseen piirretty säde ovat kohtisuorassa.
§ Tangentit ovat säteen etäisyydellä ympyrän keskipisteestä.
§ Suoran ulkopuolisesta pisteestä piirretyt kaksi tangenttia ovat yhtä pitkiä.
}
Vinkki: Yleisyyttä menettämättä voidaan ympyrä asettaa origokeskiseksi ja yksisäteiseksi, ja tangentti valita kulkemaan sopivan pisteen kautta.
\begin{vastaus}
Vinkki on hyvä
\end{vastaus}
\end{tehtava}

\begin{tehtava}
\alakohdat{
§ Määritä ympyrän $(y-r)^2+x^2 = r^2$ pisteen $(1,0)$ kautta kulkevat tangentit.
§ Minkä käyrän muodostavat vastaavat sivuamispisteet, kun $r$ saa kaikki positiiviset kokonaislukuarvot.
}
	\begin{vastaus}
		\alakohdat{
			§ Suorat $y = 0$ ja $2rx-(r^2-1)y-2r = 0$.
			§
		}
	\end{vastaus}
\end{tehtava}


%%%Sarjassamme liian helppoja todistuksia:
\begin{tehtava}
Edellä esimerkissä todettiin, että suoralla ja ympyrällä voi olla nolla, yksi tai kaksi leikkauspistettä riippuen siitä, montako ratkaisua on toisen asteen yhtälöllä joka syntyy kun kirjoitetaan ympyrän ja suoran yhtälöt yhtälöpariksi. Totea, että \emph{minkä tahansa} ympyrän ja suoran yhtälöistä todellakin aina saadaan toisen asteen yhtälö.
\begin{vastaus}
Ympyrän ja suoran yleiset yhtälöt.
\end{vastaus}
\end{tehtava}

\begin{tehtava}
Määritä ympyröiden $x^2+y^2 = 4$ ja $x^2+(y-4)^2 = 1$ yhteiset tangentit.

\begin{vastaus}
$\pm\sqrt{15}x+y-8= 0$ ja $\pm\sqrt{7}x+3y-8$. Vinkki: Hyödynnä pisteen etäisyys suorasta -kaavaa.
\end{vastaus}
\end{tehtava}

\begin{tehtava}
Määritä kaikkien niiden janojen keskipisteiden joukko, joiden toinen päätepiste on suoralla $x = 2$, toinen ympyrällä $x^2+y^2 = 2$, ja jotka kulkevat (tai niiden jatke kulkee) origon kautta. (Käyrä tunnetaan nimellä \textit{Conchoid of Nicomedes})
	\begin{vastaus}
		$4(x-1)^2(x^2+y^2) = x^2$
	\end{vastaus}
	
\end{tehtava}


\end{tehtavasivu}