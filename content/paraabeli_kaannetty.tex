\section{Vasemmalle ja oikealle aukeavat paraabelit}

\laatikko{
KIRJOITA TÄHÄN LUKUUN

\luettelo
§ muotoa $x=ay^2+by+c$ olevat paraabelit aukeavat oikealle tai vasemmalle
\loppu

KIITOS!}

%Määritelmä

%FIXME Sopisiko tähän kuva jossa taval. ja vaakasuunt. paraabeleiden symmetria-akselit?

Edellisissä kappaleissa käsiteltiin muotoa $y = ax^2 + b + c$ olevia paraabeleita, jotka ovat symmetrisiä pystyakselin suhteen (eli akselin oikea puoli on vasemman puolen peilikuva). Vaihtamalla paraabelin yhtälössä $x$- ja $y$-muuttujat keskenään saadaan yhtälö
\[x=ay^2+by+c.\]
Sen kuvaaja on joko oikealle tai vasemmalle aukeava paraabeli, jonka symmetria-akseli on vaakasuora. Kun $a>0$ paraabeli aukeaa oikealle, ja kun $a < 0$ paraabeli aukeaa vasemmalle.

%FIXME Mikä on vakiintunut suomenkielinen terminologia? 
%FIXME Tämän laatikko asettelee itsensä(?) rumasti seuraavalle sivulle, jättäen edel. sivulle ison tyhjän tilan
%Tässä ja alla käytetty samaa "oikealle tai vasemmalla aukeava (...)" -rakennetta kuin otsikossa
\laatikko{Yhtälö \[x=ay^2+by+c\] määrittelee paraabelin, joka aukeaa oikealle tai vasemmalle.}

%FIXME mainitse kuvien paraabelit tekstissä?
%Kuva: oikealle aukeava paraabeli
\begin{kuva}
    kuvaaja.pohja(-3, 4, -2, 4)
    kuvaaja.piirraParametri("0.5*t**2 -t - 1", "t", a=-2, b=4, nimi = r"$x = \frac{1}{2}y^2  - y -1$", kohta = (0.5, 1.5))
\end{kuva}

%Kuva: vasemmalle aukeava paraabeli

\begin{kuva}
    kuvaaja.pohja(-4, 3, -4, 2)
    kuvaaja.piirraParametri("-t**2 -2*t + 1", "t", a=-4, b=2, nimi = r"$x = -y^2 - 2y +1$", kohta = (0.3, 1.2))
\end{kuva}

%FIXME Perusesimerkki tähän

%Käännettyjen(???) paraabeleiden ominaisuuksia

Vasemmalle ja oikealle aukeaville paraabeleille pätevät vastaavat tulokset kuin aikaisemmin käsitellyille pystysuorille paraabeleille. Esimerkiksi vaakasuuntaisen paraabelin huippu sijaitsee aina pisteessä

\[y = \frac{-b}{2a}.\]

%Huippumuoto käännetylle paraabelille

Myös paraabelille, jonka yhtälö on $x=ay^2+by+c$ voidaan kirjoittaa huippumuotoinen yhtälö:

\laatikko{Oikealle tai vasemmalle aukeavan paraabelin huippumuotoinen yhtälö on
\[
x-x_0 = a(y-y_0)^2.
\]}

%FIXME Tähän tehtäväesimerkki ylläolevasta
\begin{esimerkki}
    Määritä sellaisen vasemmalle tai oikealle aukeavan paraabelin yhtälö, jonka huippu on samassa pisteessä kuin paraabelin $y = x^2 + 4x +1$ huippu ja myös leikkaa $y$-akselin samassa pisteessä kuin kyseinen paraabeli.
    \begin{esimratk} % ratkaisu
        Etsitään ensiksi paraabelin $y = x^2 + 4x +1$ huippu. Voisimme täydentää paraabelin neliöksi, mutta helpompaa on käyttää aikaisemmin johtamaamme tulosta (pystysuuntaisen) paraabelin huipun $x$-koordinaatille
        \begin{align*}
        x &= \frac{-b}{2a} \\
          &= \frac{-4}{2\,\cdot\,1} = -2, \\
        \intertext{minkä jälkeen saamme huipun $y$-koordinaatin sijoittamalla}
        y &= x^2 + 4x +1 = (-2)^2 - 8 +1 = -3.
        \end{align*}
        Huipun koordinaatit ovat siis $(-2, -3)$. Tämä on myös kysytyn paraabelin huippu, joten sijoitetaan se vasemmalle tai oikealle aukeavan paraabelin huippumuotoiseen yhtälöön.
        \begin{align*}
        x-x_0 &= a(y-y_0)^2\\
        x-(-2) &= a(y-(-3))^2 \\
        x +2 &= a(y^2 +6y +9)\\
        x &= ay^2 +6ay +9a -2
        \end{align*}
        Vielä on selvitettävä kerroin $a$. Tehtävänannon mukaan kysytty paraabeli leikkaa $y$-akselin samassa pisteessä kuin $y = x^2 + 4x +1$, selvitetään siis tuo piste. Tutkimalla koordinaatistoa huomataan, että $y$-akseli on itse asiassa suora $x = 0$, mikä on siten leikkauspisteen $x$-koordinaatti. Sijoittamalla tämän paraabelin yhtälöön saamme $y$-koordinaatin:
        \begin{align*}
        y &= x^2 + 4x +1\\
        y &= 0 + 0 + 1 = 1
        \end{align*}
        Näin ollen molemmat paraabelit kulkevat pisteen $(0,1)$ kautta. Sijoittamalla tämä kysytyn paraabelin yhtälöön voimme ratkaista tuntemattoman vakion $a$ arvon.
        \begin{align*}
        x &= ay^2 +6ay +9a -2 && \ppalkki (x,y) = (0,1) \\
        0 &= a +6a +9a -2 \\
        2 = 16a \\
        a = \frac{1}{8}\\
        \intertext{Tästä saamme vaakasuuntaan (oikealle) aukeavan paraabelin yhtälöksi:}
        x &= ay^2 +6ay +9a -2 \\
        x &= \frac{1}{8}y^2 + \frac{6}{8}y + \frac{9}{8} -2  && \ppalkki \frac{9}{8} -2 = \frac{9 - 16}{8}\\
        x &= \frac{1}{8}y^2 + \frac{3}{4}y - \frac{7}{8}
        \end{align*}
    \end{esimratk}
    \begin{esimvast} % vastaus
        Paraabelin yhtälö on $x = \frac{1}{8}y^2 + \frac{3}{4}y - \frac{7}{8}$.
        %%%FIXME: KUVA!
        %%https://www.wolframalpha.com/input/?i=y+%3D+x%5E2++%2B+4x++%2B1%2C+x+%3D+1%2F8%28y%5E2+%2B+6y+-7%29
    \end{esimvast}
\end{esimerkki}

%Muita esimerkkejä?

%FIXME Paraabelin yleinen muoto (johtosuora ax +by +c = 0) (kartioleikkaukset??) liitteisiin?

%Tehtävät:

\begin{tehtavasivu}

\subsubsection*{Opi perusteet}

\begin{tehtava}
    Mihin suuntaan aukeaa paraabeli, jonka yhtälö on
    \alakohdat
        § $x = -2y^2 + y + 1$
        § $x = y^2 - 3y -2$
        § $x = 4y^2 - 1y +3$
    \loppu
    \begin{vastaus}
          \alakohdat
          § Vasemmalle.
          § Oikealle.
          § Oikealle.
          \loppu
    \end{vastaus}
\end{tehtava}

\begin{tehtava}
    Hahmottele edellisen tehtävän paraabelit ja laske niiden huippujen koordinaatit.
    \begin{vastaus}
          Huiput sijaitsevat pisteissä $\left(\frac{9}{8}, \frac{1}{4}\right)$, $\left(\frac{17}{4},\frac{3}{2}\right)$ ja $\left(\frac{47}{16},\frac{1}{8}\right).$
    \end{vastaus}
\end{tehtava}

\subsubsection*{Hallitse kokonaisuus}

\subsubsection*{Sekalaisia tehtäviä}

TÄHÄN TEHTÄVIÄ SIJOITTAMISTA ODOTTAMAAN


\end{tehtavasivu}