\section{Ympyrä}

\laatikko{
KIRJOITA TÄHÄN LUKUUN

\luettelo{
§ ...
}

KIITOS!}

\begin{kuva}
    kuvaaja.pohja(-3.5, 3.5, -3.5, 3.5, korkeus = 4, nimiX = "$x$", nimiY = "$y$", ruudukko = True)
    kuvaaja.piirraParametri("3*cos(t)", "3*sin(t)", a = 0, b = 2*pi)
    piste((3*cos(0.75), 3*sin(0.75)), "(x, y)", -120)
\end{kuva}

Kuvaan on piirretty käyrä, jonka pisteiden etäisyys origosta on 3. Käyrän huomataan olevan ($3$-säteinen) ympyrä. Piste $(x, y)$ on tällä ympyrällä täsmälleen silloin, kun sen etäisyys origosta on 3. Toisin sanoen täytyy päteä $\sqrt{x^2+y^2}=3$. Kun yhtälön molemmat puolet korotetaan vielä toiseen potenssiin, saadaan $x^2+y^2=9$. $3$-säteisen ympyrän yhtälö on siis $x^2+y^2=9$.

\laatikko{\termi{ympyrä}{Ympyrä} muodostuu tason pisteistä, jotka ovat vakioetäisyydellä jostakin kiinteästä pisteestä. Tätä kiinteää pistettä kutsutaan ympyrän \termi{keskipiste}{keskipisteeksi} ja vakioetäisyyttä ympyrän \termi{säde}{säteeksi}.}

Johdetaan yhtälö ympyrälle, jonka keskipiste on $(x_0, y_0)$ ja säde $r$. Piste $(x, y)$ on ympyrälllä täsmälleen silloin, jos sen etäisyys pisteestä $(x_0, y_0)$ on $r$. Pisteiden välisen etäisyyden kaavalla saadaan pisteiden $(x, y)$ ja $(x_0, y_0)$ väliseksi etäisyydeksi $\sqrt{(x-x_0)^2+(y-y_0)^2}$. Tuloksena on siis yhtälö
\[
\sqrt{(x-x_0)^2+(y-y_0)^2}=r.
\]
Koska säde $r$ ei voi olla negatiivinen, voidaan yhtälön molemmat puolet korottaa toiseen potenssiin ja saadaan yhtäpitävä yhtälö
\[
(x-x_0)^2+(y-y_0)^2=r^2.
\]

Erityisesti, jos ympyrän keskipiste on origo eli $(x_{0}, y_{0})= (0, 0)$, yhtälö saa muodon

\[
x^{2}+y^{2} = r^{2}.
\]

\laatikko{
Jos ympyrän keskipiste on $(x_{0}, y_{0})$ ja säde $r > 0$, ympyrän yhtälö on
\[
(x-x_{0})^{2}+(y-y_{0})^{2} = r^{2}.
\]
}

Jos säde $r$ on nolla, yhtälön toteuttaa vain piste $(x_{0}, y_{0})$. Tällöin kyseessä on vain yksi piste.


%%% FIX ME Eikö tämä ja alempi kommentti ole aivan pätevä esimerkki + yksikköympyrän määritelmä?
%\begin{esimerkki}
%Ympyrän keskipiste on $(-4, 1)$ ja säde $5$. Määritä ympyrän yhtälö ja hahmottele ympyrä koordinaatistoon.
%\begin{esimratk}
%Ympyrän yhtälö saadaan käyttämällä edellä annettua kaavaa. Nyt $x_0=-4$, $y_0=1$ ja $r=5$. Ympyrän yhtälöksi saadaan $(x-(-4))^2+(y-1)^2=25$ eli
%\[
%(x+4)^2+(y-1)^2=25.
%\]
%\end{esimratk}
%\begin{esimvast}
%Ympyrän yhtälö on $(x+4)^2+(y-1)^2=25$.
%\end{esimvast}
%\end{esimerkki}

%Tähän kuva ympyrästä.

%Edellisen esimerkin ympyrän yhtälö voidaan kirjoittaa myös toisenlaisessa muodossa.
%\begin{align*}
%(x+4)^2+(y-1)^2&=25 \\
%x^2+8x+16+y^2-2y+1&=25 \\
%x^2+y^2+8x-2y-8&=0.
%\end{align*}

\begin{esimerkki}
Piirrä kuva ympyrästä, jonka yhtälö on
\[
(x-1)^2+(x+1)^2=4.
\]
\begin{esimratk}
Muokataan ympyrän yhtälöä niin, että keskipiste ja säde näkyvät suoraan:
\[
(x-1)^2+(x-(-1))^2=2^2.
\]
Tästä nähdään, että keskipiste on $(1, -1)$ ja säde 2. Nyt kuva on helppo piirtää.

\begin{kuva}
    kuvaaja.pohja(-2, 4, -4, 2, korkeus = 4, nimiX = "$x$", nimiY = "$y$", ruudukko = True)
    kuvaaja.piirraParametri("2*cos(t)+1","2*sin(t)-1", a = 0, b = 2*pi)
\end{kuva}

\end{esimratk}
\end{esimerkki}

%\begin{esimerkki}
%Ympyrän $\Gamma_{1}$\footnote{sdf} keskipiste on $(3, -4)$ ja säde $\sqrt{2}$, ja ympyrän $\Gamma_{2}$ keskipiste origo ja säde 1. Määritä ympyröiden yhtälöt. Hahmottele ympyrät koordinaatistoon.

%\begin{esimratk}
%Edellisen mukaan ympyrän $\Gamma_{1}$ yhtälö on
%\[
%(x-3)^{2}+(y-(-4))^{2} = (\sqrt{2})^{2}
%\]
%eli sievennettynä
%\[
%(x-3)^{2}+(y+4)^{2} = 2.
%\]
%$\Gamma_{2}$:n yhtälö saadaan vastaavasti:
%\[
%x^{2}+y^{2} = 1.
%\]
%Edelliselle yksikkösäteiselle origokeskiselle ympyrälle on vakiintunut nimitys \emph{yksikköympyrä}.

%\end{esimratk}
%\end{esimerkki}

%$(x_0, y_0)$-keskisen, $r$-säteisen ympyrän yhtälön muuntaminen normaalimuotoon:
%\begin{align*}
%(x - x_0)^2 + (y - y_0)^2 = r^2 && \ppalkki \textnormal{Käytetään 		%binomimuistikaavoja.}\\
%x^2 - 2x_0 x + x_0^2 + y^2 - 2y_0 y + y_0^2 = r^2 && \ppalkki -r^2\\
%x^2 + y^2 - 2x_0 x - 2y_0 y - r^2 + x_0^2 + y_0^2 = 0 && \\
%\end{align*}

Aina keskipiste ja säde eivät näy ympyrän yhtälöstä suoraan.
Esimerkiksi yhtälö
\[
x^2+6x+y^2-4y=3
\]
on erään ympyrän yhtälö.
Tämä nähdään täydentämällä summattavat $x^2+6x$ ja $y^2-4y$ neliöiksi.

Neliöksi täydentäminen opittiin kurssissa MAA2, mutta kerrataan se tässä vielä.
Aloitetaan lausekkeesta $x^2+6x$. Se on lähes sama kuin binomin $x+3$ neliö, sillä $(x+3)^2=x^2+6x+9$. Ainoastaan vakiotermit poikkeavat toisistaan ja tämän voi korvata lisäämällä ympyrän yhtälön molemmille puolille luvun $9$: 
\begin{align*}
x^2+6x+y^2-4y &= 3 && \ppalkki +9 \\
(x^2+6x+9)+(y^2-4y) &= 12 && \\
(x+3)^2+(y^2-4y) &= 12.&& 
\end{align*}

Siirrytään sitten tarkastelemaan summattavaa $y^2-4y$. Se on puolestaan melkein binomin $y-2$ neliö, sillä $(y-2)^2=y^2-4y+4$. Binomin neliö saadaan näkyviin lisäämällä yhtälön molemmille puolille luku $4$:
\begin{align*}
(x+3)^2+(y^2-4y) &= 12 && \ppalkki +4\\
(x+3)^2+(y^2-4y+4) &= 16 && \\
(x+3)^2+(y-2)^2 &= 16 && \\
(x+3)^2+(y-2)^2 &= 4^2.&& 
\end{align*}

Nyt huomataan, että kyseessä on $(-3, 2)$-keskisen $4$-säteisen ympyrän yhtälö.

\begin{esimerkki}
Ympyrän yhtälö on $x^2-8x+y^2+5y+3=0$. Määritä ympyrän keskipiste ja säde.
\begin{esimratk}
Yhtälö muuttuu muotoon $x^2-8x+y^2+5y=-3$. Suoritetaan sitten neliöksi täydentäminen:
\begin{align*}
x^2-8x+y^2+5y&=-3 && \ppalkki +16\\
x^2-8x+16+y^2+5y&=-3+16 && \\
(x^2-4)^2+y^2+5y&=13 && \ppalkki +\frac{25}{4}\\
(x^2-4)^2+y^2+5y+\frac{25}{4}&=\frac{77}{4} && \ppalkki +\frac{25}{4}\\
(x^2-4)^2+\left(y+\frac{5}{4}\right)^2&=\frac{102}{4} && \\
(x^2-4)^2+\left(y+\frac{5}{4}\right)^2&=\frac{51}{2} && 
\end{align*}
Nähdään, että keskipiste on $(4, -5/4)$ ja säde $\sqrt{51/2}$.
\end{esimratk}
\begin{esimvast}
Keskipiste on $(4, -5/4)$ ja säde $\sqrt{51/2}$.
\end{esimvast}
\end{esimerkki}

\begin{esimerkki}
Onko yhtälö $x^2-4x+y^2+2y+6=0$ ympyrän yhtälö?
\begin{esimratk}
Suoritetaan neliöksitäydennys:
\begin{align*}
x^2-4x+y^2+2y+6&=0 && \ppalkki -6\\
x^2-4x+y^2+2y&=-6 && \ppalkki +4\\
x^2-4x+4+y^2+2y&=-2 && \\
(x^2-2)^2+y^2+2y&=-2 && \ppalkki +1\\
(x^2-2)^2+y^2+2y+1&=-1 && \\
(x^2-2)^2+(y+1)^2&=-1. &&
\end{align*}
Nyt nähdään, että kyseessä ei voi olla ympyrän yhtälö, sillä säteen neliö ei voi olla negatiivinen.
\end{esimratk}
\begin{esimvast}
Kyseessä ei ole ympyrän yhtälö.
\end{esimvast}
\end{esimerkki}

Edellä tehty neliöön korotus voidaan voidaan suorittaa yleisesti muotoa
\[
x^2+ax+y^2+by+c = 0
\]
oleville yhtälöille. Täydentämällä $x^2+ax$ ja $y^2+by$ neliöiksi saadaan
\begin{align*}
x^2+ax+y^2+by+c &= 0 && \ppalkki +\frac{a^2}{4}+\frac{b^2}{4}-c \\
\Big(x^2+ax+\frac{a^2}{4}\Big)+\Big(y^2+by+\frac{b^2}{4}\Big) &= \frac{a^2}{4}+\frac{b^2}{4}-c  \\
\Big(x+\frac{a}{2}\Big)^2+\Big(y+\frac{b}{2}\Big)^2 &= \frac{a^2}{4}+\frac{b^2}{4}-c
\end{align*}
Jos yhtälön oikea puoli eli $\frac{a^2}{4}+\frac{b^2}{4}-c \geq 0$ yhtälö kuvaa $\sqrt{\frac{a^2}{4}+\frac{b^2}{4}-c}$-säteistä $(-\frac{a}{2}, -\frac{b}{2})$-keskistä ympyrää. Jos oikea puoli on nolla, yhtälö kuvaa vastaavaa pistettä. Jos se on negatiivinen, yhtälö ei kuvaa mitään käyrää: yhtälön vasen puoli on (kahden neliön summana) aina suurempi tai yhtä suuri kuin nolla, joten oikean puolen ollessa negatiivinen mikään reaalilukupari ei toteuta yhtälöä.

%\laatikko{Yhtälöt muotoa
%\[
%x^2+ax+y^2+by+c = 0
%\]
%kuvaavat ympyrää, pistettä tai tyhjää joukkoa.}

\begin{tehtavasivu}

\paragraph*{Opi perusteet}

\begin{tehtava}
	Ympyrän keskipiste on $(0, 0)$ ja säde $5$. Muodosta ympyrän yhtälö.
	\begin{vastaus}
		$x^2+y^2=25$
	\end{vastaus}
\end{tehtava}

\begin{tehtava}
	Ympyrän keskipiste on $(1, -4)$ ja säde $10$. Muodosta ympyrän yhtälö.
	\begin{vastaus}
		$(x-1)^2+(y+4)^2=100$ tai auki kirjoitettuna $x^2+y^2-2x+8y-83=0$ % auki kirjoitettu vai aukikirjoitettu
	\end{vastaus}
\end{tehtava}

\begin{tehtava}
	Määritä keskipiste ja säde.
	\alakohdat{
		§ $(x-3)^2+(y+7)^2=12$
		§ $x^2+y^2=49$
	}
	\begin{vastaus}
		\alakohdat{
			§ keskipiste $(3, -7)$, säde $2\sqrt{3}$
			§ keskipiste $(0, 0)$, säde $7$
		}
	\end{vastaus}
\end{tehtava}

\begin{tehtava}
	Määritä keskipiste ja säde.
	\alakohdat{
		§ $x^2+y^2-10x+16y+72=0$
		§ $x^2+y^2+8x-22y+129=0$
	}
	\begin{vastaus}
		\alakohdat{
			§ keskipiste $(5, -8)$, säde $\sqrt{17}$
			§ keskipiste $(-4, 11)$, säde $2\sqrt{2}$
		}
	\end{vastaus}
\end{tehtava}

\begin{tehtava}
	Määritä ympyrän $(x+10)^2+y^2=2$ keskipiste ja säde ja ratkaise ympyrän yhtälöstä $y$. 
	\begin{vastaus}
		keskipiste $(-10, 0)$, säde $\sqrt{2}$, $y=\pm\sqrt{2-(x+10)^2}$ 
	\end{vastaus}
\end{tehtava}

\begin{tehtava}
	Ympyrän keskipiste on origo ja säde $3$. Mitkä seuraavista pisteistä ovat ympyrän kehällä?
	\alakohdat{
		§ $(10, -2)$
		§ $(-3, 0)$
		§ $(2, \sqrt{5})$
		§ $(1, 3)$
	}
	\begin{vastaus}
		b) ja c)
	\end{vastaus}
\end{tehtava}

\begin{tehtava}
	Määritä $k$ niin, että lauseke $(x-3)^2+(y+3)^2=k$ on
	\alakohdat{
			§ ympyrä
			§ $\sqrt{7}$-säteinen ympyrä
			§ origon kautta kulkeva ympyrä?
	}
	\begin{vastaus}
		\alakohdat{
			§ $k>0$
			§ $k=7$
			§ $k=18$
		}
	\end{vastaus}
\end{tehtava}

\begin{tehtava}
	Tutki, mitä yhtälöiden kuvaajat esittävät.
	\alakohdat{
		§ $x^2+y^2-6x+4y+4=0$
		§ $x^2+y^2+14x-6y+10=0$
	}
	\begin{vastaus}
		\alakohdat{
			§ ympyrä
			§ piste
		}
	\end{vastaus}
\end{tehtava}

\paragraph*{Hallitse kokonaisuus}

\begin{tehtava}
	Määritä ympyrän keskipiste ja säde.
	\alakohdat{
		§ $(x+t)^2+(y+u)^2=k, k>0$
		§ $(x+2)^2+(y-7)^2=-8$
	}
	\begin{vastaus}
		\alakohdat{
			§ keskipiste $(-t, -u)$, säde  $\sqrt{k}$
			§ ei ole ympyrä
		}
	\end{vastaus}
\end{tehtava}

\begin{tehtava}
	Ympyrä sivuaa $y$-akselia pisteessä $(0, -1)$ ja kulkee pisteen $(3, 2)$ kautta. Mikä on ympyrän yhtälö?
	\begin{vastaus}
		$(x-3)^2+(y+1)^2=9$
	\end{vastaus}
\end{tehtava}

\begin{tehtava}
	Ympyrä kulkee pisteiden $(1, 6), (-2, 5)$ ja $(5, 4)$ kautta. Mikä on ympyrän yhtälö?
	\begin{vastaus}
		$(x-1)^2+(y-1)^2=16$
	\end{vastaus}
\end{tehtava}

\begin{tehtava}
	Jana, jonka pituus on $t$ liikkuu koordinaatistossa siten, että sen toinen pää on $x$-akselilla ja toinen $y$-akselilla. Mitä käyrää pitkin liikkuu janan keskipiste?
	\begin{vastaus}
		$x^2+y^2=\frac{1}{4}t^2$
	\end{vastaus}
\end{tehtava}

\begin{tehtava}
	Millä $c$:n reaaliarvoilla yhtälö $x^2-2xc+y^2-2c-2 = 0$ esittää ympyrää? Mikä on tällöin ympyrän keskipiste ja säde? Todista, että ympyrä kulkee tällöin kahden $c$:stä riippumattoman pisteen kautta.
	\begin{vastaus}
		Kaikilla, keskipiste $(c,0)$, säde $\sqrt{(c-1)^2+1}$. Ympyrät kulkevat aina pisteiden $(1,\pm 1)$ kautta.
	\end{vastaus}
\end{tehtava}

\begin{tehtava}
	Ympyrä voidaan määritellä myös monella muulla yhtäpitävällä tavalla
	\alakohdat{
		§ Jos $A = (1,0)$ ja $B = (-1,0)$, määritä kaikki ne pisteet $P$, joilla $AP$ ja $BP$ ovat kohtisuorassa.
		§ Jos $A = (2,0)$ ja $B = (-1,0)$, määritä kaikki ne pisteet $P$, joille $\frac{AP}{BP} = 2$.
		§ Jos $A = (3,0)$ ja $B = (-1,0)$, määritä kaikki ne pisteet $P$, joille $AP^2+BP^2 = 10$.
	}
	\begin{vastaus}
		\alakohdat{
			§ Ympyrän $x^2+y^2 = 1$ pisteet (lukuunottamatta pisteitä $A$ ja $B$)
			§ Ympyrän $(x+2)^2+y^2 = 4$ pisteet
			§ Ympyrän $(x-1)^2+y^2 = 1$ pisteet
		}
	\end{vastaus}
\end{tehtava}

\paragraph*{Sekalaisia tehtäviä}

TÄHÄN TEHTÄVIÄ SIJOITTAMISTA ODOTTAMAAN







\end{tehtavasivu}
