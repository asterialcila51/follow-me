\section{Tehtäviä ylioppilaskokeista}

%%Pitäisikö tämän olla TEHT_ylioppilaskokeet kuten MAA1 ja MAA2? Nyt esim vastaukset ei toimi.
%%Pitäisikö tämän olla TEHT_ylioppilaskokeet kuten MAA1 ja MAA2?
%%Pitäisikö tämän olla TEHT_ylioppilaskokeet kuten MAA1 ja MAA2?
%%Pitäisikö tämän olla TEHT_ylioppilaskokeet kuten MAA1 ja MAA2?

\subsubsection*{Lyhyen oppimäärän tehtäviä}

\begin{tehtava}
Määrää $a$ siten, että yhtälön $x^2-2x+a=0$ toinen juuri tulee yhtäsuureksi kuin toisen juuren neliö. Mitkä ovat juuret? (K1941/3)
\end{tehtava}
%Lisännyt Aleksi Sipola 17.5.2014

\subsubsection*{Pitkän oppimäärän tehtäviä}

\begin{tehtava}
	Muodosta sen suoran yhtälö, joka kulkee pisteiden $(4, -3)$ ja $(-2,6)$ kautta. (S07/1b)
\end{tehtava}

\begin{tehtava}
	Missä pisteessä suora $y=3x-4$ leikkaa x-akselia? (K06/1b)
\end{tehtava}

\begin{tehtava}
	Määritä ympyrän $x^2+y^2+4x-2y+1=0$ niiden tangettien yhtälöt, jotka kulkevat pisteen $(1,3)$ kautta. (S07/5)
\end{tehtava}



\begin{tehtava}
Ympyrä sivuaa suoraa $3x-4y=0$ pisteessä $(8,6)$. Lisäksi se sivuaa positiivista $x$-akselia.
Määritä ympyrän keskipiste ja säde. (K2014/5)
\end{tehtava}
%Lisännyt Aleksi Sipola 17.5.2014
 % \begin{vastaus}
%	Ympyrän keskipiste on $(10,frac[10][3])$ ja $r=30$
 %   \end{vastaus}


\begin{tehtava}
Taso $x+2y+3z=6$ leikkaa positiiviset koordinaattiakselit pisteissä $A$, $B$ ja $C$.
a)Määritä sen tetraedrin tilavuus, jonka kärjet ovat origossa $O$ sekä pisteissä $A$,$B$ ja $C$. (K2014/9)
%  \begin{vastaus} VASTAUKSET EIVÄT MENE VASTAUS OSIOON VAAN JÄÄVÄT TEHTÄVÄNANNON PERÄÄN KOSKA LIITE
%	 Tilavuus on $6$
%  \end{vastaus}
\end{tehtava}
%Lisännyt Aleksi Sipola 17.5.2014


\begin{tehtava}
Pöydällä on kolme samankokoista palloa, joista kukin koskettaa kahta muuta. Niiden päälle asetetaan neljäs samanlainen  pallo, joka koskettaa kaikkia kolmea alkuperäistä palloa.

Mikä on rakennelman korkeus? Anna vastauksena tarkka arvo pallojen säteen avulla lausuttuna. (K2013/10)
\end{tehtava}
%Lisännyt Aleksi Sipola 17.5.2014


\begin{tehtava}
Tarkastellaan tasokäyrää, jonka yhtälö on $2x^2+2y^2-3xy-2x+2y-4=0$
\begin{alakohdat}
		\alakohta{Määritä käyrän ja koordinaattiakselien leikkauspisteet. (2p.)}
		\alakohta{Osoita, että kaikki leikkauspisteet ovat saman ympyrän kehällä, ja määritä tämän ympyrän yhtälö. (3p.)}
		\alakohta{Suora kulkee origon ja b-kohdan ympyrän keskipisteen kautta. Missä pisteissä tämä suora leikkaa alkuperäisen käyrän? (2p.)}
		\alakohta{Onko alkuperäinen käyrä ympyrä? (2p.)}
	\end{alakohdat}
 (K2013/*14)
\end{tehtava}
%Lisännyt Aleksi Sipola 17.5.2014


\begin{tehtava}
Pöydällä on kolme samankokoista palloa, joista kukin koskettaa kahta muuta. Niiden päälle asetetaan neljäs samanlainen  pallo, joka koskettaa kaikkia kolmea alkuperäistä palloa.

Mikä on rakennelman korkeus? Anna vastauksena tarkka arvo pallojen säteen avulla lausuttuna. (K2013/10)
\end{tehtava}
%Lisännyt Aleksi Sipola 17.5.2014

\begin{tehtava} KUVA TARVITAAN Ratkaise 
\begin{alakohdat}
		\alakohta{Kaksi ympyrää sivuaa toisiaan ja $x$-akselia kuvan 1 mukaisesti. Määritä ympyröiden keskipisteiden vaakasuora etäisyys $d$ niiden säteiden avulla lausuttuna. (3 p.)}
		\alakohta{Kolme ympyrää sivuaa toisiaan ja $x$-akselia kuvan 2 mukaisesti. Määritä keskimmäisen ympyrän säde $r_3$ kahden reunimmaisen ympyrän säteiden avulla lausuttuna. (3 p.)}
		\alakohta{Todista René Descartesin (1596-1650) keksimä b-kohdan ympyröihin liittyvä kaava
		
		$(k_1+k_2+k_3)^2=2(k_1^2+k_2^2+k_3^2)$, \\
		jossa $k_i=\frac{1}{r_i}$,  $i=1,2,3.$ (3 p.)}	
	\end{alakohdat}
	(K2012/*15)
\end{tehtava}
%Lisännyt Aleksi Sipola 17.5.2014

\begin{tehtava} Ratkaise
  \begin{alakohdat}
		\alakohta{$\frac{2}{x}=\frac{3}{x-2}$}
		\alakohta{$x^2-2\leq x$}
		\alakohta{$\left|\frac{3}{2}x-6\right|=6$}	
  \end{alakohdat}
	(K2011/1)
\end{tehtava}
%Lisännyt Aleksi Sipola 17.5.2014

\begin{tehtava}
Puolipallon sisällä on kuutio siten, että sen yksi sivutahko on puolipallon pohjatasolla ja vastakkaisen sivutahkon kärkipisteet ovat pallopinnalla. Kuinka monta prosenttia kuution tilavuus on puolipallon tilavuudesta?
(K2010/4)
\end{tehtava}

\begin{tehtava}
Tietunnelin poikkileikkaus on osa alaspäin aukeavaa paraabelia. Tien leveys on $10 m$, ja tunnelin poikkileikkauksen pinta-ala on $25 m^2$. Määritä tunnelin korkeus senttimetrin tarkkuudella.
(K2010/8)
\end{tehtava}

\begin{tehtava}
Ratkaise epäyhtälö
$\frac{-x^2+x+2}{x^3+2x^2-3x}>0$
(K2009/8)
\end{tehtava}

\begin{tehtava}
Mikä paraabelin $y=5-x^2$ piste on lähinnä origoa? Piirrä kuvio.
(K2009/9)
\end{tehtava}



\begin{tehtava} Tee nämä.
  \begin{alakohdat}
		\alakohta{Määritä suorien $2x+y=8$ ja $3x+2y=5$ leikkauspiste}
		\alakohta{Ratkaise yhtälö $|3x-2|=5$}	
  \end{alakohdat}
	(K2008/2 a ja c)
\end{tehtava}
%Lisännyt Aleksi Sipola 17.5.2014


\begin{tehtava}
Laske paraabelien $y=x^2-3$ ja $y=-x^2+2x+1$ leikkauspisteiden koordinaatit.(K2008/7a)
\end{tehtava}
%Lisännyt Aleksi Sipola 17.5.2014

\begin{tehtava}
Ympyrä ja piste sen ulkopuolella ovat tunnetut. Hae pistettä ja ympyrän keskipistettä yhdistävällä suoralla viivalla semmoinen piste jonka etäisyys tunnetusta pisteestä on yhtäsuuri kuin ne tangentit, jotka haettavasta pisteestä saatetaan piirtää ympyrälle. (K1983/7)
\end{tehtava}
%Lisännyt Aleksi Sipola 17.5.2014

\begin{tehtava}
Määrää $a$ siten, että yhtälön $x^2+(a+2)x-a^2=0$ suuremman ja pienemmän juuren eroitus saa mahdollisimman pienen arvon. Mitkä ovat juuret? (K1941/3)
\end{tehtava}
%Lisännyt Aleksi Sipola 17.5.2014


\begin{tehtava}
Ratkaise ekvationit:
\[
\left\{
\begin{aligned}
 x+\frac{1}{2}(y+z)=102    \\
 y+\frac{1}{2}(x+z)=78  \\
 z+\frac{1}{2}(x+y)=61
\end{aligned}
\right. 
\]
(S1893/3)
\end{tehtava}
%Lisännyt Aleksi Sipola 17.5.2014

