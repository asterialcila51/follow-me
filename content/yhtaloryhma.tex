\section{Lineaariset yhtälöryhmät} 

\laatikko{
KIRJOITA TÄHÄN LUKUUN

\begin{itemize}
\item mikä yhtälöryhmä on
\item miten ratkaistaan yhtälöpari (sijoitus, yhteenlaskumenetelmä)
\item että ratkaisuja voi olla yksi, nolla tai äärettömän monta
\item miten useamman tuntemattoman yhtälöryhmä ratkaistaan
\end{itemize}

KIITOS!}

\subsection*{Johdanto: Lineaarinen yhtälö}

Kursseilla MAA1 ja MAA2 on tutustuttu ensimmäisen 
\[ax +b = 0\]
ja toisen asteen yhtälöihin
\[ax^2 + bx +c = 0,\]
jotka molemmat ovat yhden muuttujan ($x$) yhtälöitä.

Edellä olemme käsitelleet suoran yhtälöä
\[ax + by + c = 0,\]
joka on kahden muuttujan, $x$:n ja $y$:n, yhtälö, jonka termit ovat ensimmäistä astetta tai vakioita.

Yleisesti tällaisiä yhtälöitä, joissa termit eivät ole ensimmäistä astetta suurempia kutsutaan \termi{lineaarinen yhtälö}{lineaariksi yhtälöiksi}.

\laatikko{
  \termi{lineaarinen yhtälö}{Lineaarisessa yhtälössä} kukin termi on korkeintaan korkeintaan ensimmäistä astetta. 
  \begin{samepage}
    Esimerkiksi
    \[ax + b = 0,\]
    \[ax + bx + c = 0\]
    ja
    \[ax + by + cz + d = 0\]
    ovat lineaarisia yhtälöitä.
  \end{samepage}
}

\subsection*{Yhtälöryhmä}

\termi{yhtälöryhmä}{Yhtälöryhmällä} tarkoitetaan useaa yhtälöä, joiden täytyy
päteä samanaikaisesti. \footnote{Englannin kielessä yhtälöryhmästä käytetäänkin usein nimeä 
  \emph{simultaneous equations} eli samanaikaiset yhtälöt.
  Termi \emph{system of equations} on tosin vähintäänkin yhtä yleinen.}
Yhtälöryhmän ratkaisujen tulee siis toteuttaa kaikki ryhmän yhtälöt; ratkaisuja ei välttämättä ole.

Tässä luvussa käsitellään yhtälöryhmiä, joissa kaikki yhtälöt ovat \emph{lineaarisia}. 
Tällaisia yhtälöryhmiä kutsutaan \termi{lineaarinen yhtälöryhmä}{lineaarisiksi yhtälöryhmiksi}.

Esimerkiksi
\[
\left\{
\begin{aligned}
3x-2y-z&= -5 \\
5x+6y+5z&= 1 \\
-x+5y&= 0.
\end{aligned}
\right.
\]
on lineaarinen yhtälöryhmä. 

%Lisäksi myöhemmin kirjassa käsitellään muutamia erikoistapauksia toisen asteen yhtälöryhmistä
%ratkaistaessa kahden ympyrän tai ympyrän ja suoran leikkauspisteitä.
% re ^:
%  Ympyrä-kappaleessa implisiittisesti? onko menetelmät jos esitelty siellä, kyllä? 
%  Jos kyllä, harkittava pitäisikö sitten siinä kohdin mainita 'ratkaistaan \termi{epälineaarinen yhtälöryhmä} blah blah...'
%  Vai oma, erillinen kappale, tyyliin 'Luku n. Joitain epälineaarisia yhtälöryhmiä', vaikkapa lisämateriaaliksi?

\subsection*{Yhtälöpari}

Yksinkertaisin mielenkiintoinen yhtälöryhmä on
\termi{lineaarinen yhtälöpari}{lineaarinen yhtälöpari}.
Lineaarisessa yhtälöparissa on kaksi ensimmäistä astetta olevaa yhtälöä.
Tällä kurssilla yhtälöparilla tarkoitetaan useimmiten kahden muuttujan lineaarista yhtälöparia.

Lineaarinen yhtälöpari voidaan esittää monella tapaa. Tässä
kirjassa käytämme pääasiallisesti muotoa, jossa kumpikin yhtälö on normaalimuodossa.

\laatikko[Kahden muuttujan yhtälöparin normaalimuoto]{
\begin{samepage}
  \[
    \left\{
      \begin{aligned}
      a_1x+b_1y+c_1 &= 0 \\
      a_2x+b_2y+c_2 &= 0,
      \end{aligned}
    \right.
  \]
missä $a_1, a_2, b_1, b_2, c_1, c_2 \in \rr$
\end{samepage}}

Kahden muuttujan lineaarisen yhtälöparin ratkaisu on lukupari $(x, y)$, joka toteuttaa molemmat yhtälöt.

Lineaarisia yhtälöpareja ratkotaan lukiomatematiikassa pääasiallisesti kahdella menetelmällä.
Nämä menetelmät ovat \termi{sijoitusmenetelmä}{sijoitusmenetelmä} ja
\termi{yhteenlaskumenetelmä}{yhteenlaskumenetelmä}.

\begin{esimerkki}
Ratkaise yhtälöpari
\[
\left\{
\begin{aligned}
3x-2y&= 1 \\
-x+5y&= 2.
\end{aligned}
\right.
\]
\begin{esimratk}
On löydettävä kaikki ne luvut $x$ ja $y$ jotka toteuttavat molemmat yhtälöt.

Käytetään \termi{sijoitusmenetelmä}{sijoitusmenetelmää}. Ratkaistaan tuntematon $x$ alemmasta yhtälöstä ja sijoitetaan se ylempään yhtälöön. Alemmasta yhtälöstä $-x+5y= 2$ saadaan $x=5y-2$. Sijoitetaan tämä ylempään yhtälöön $3x-2y=1$:
\begin{align*}
3x-2y&=1 && \ppalkki \text{Sijoitetaan $x=5y-2$.} \\
3(5y-2)-2y&=1 && \\
15y-6-2y&=1 && \\
13y&=7 && \\
y&=\frac{13}{7} && \\
\end{align*}
Sijoitetaan $y=13/7$ alempaan yhtälöön, jotta voidaan ratkaista $x$:
\begin{align*}
-x+5y&= 2 && \ppalkki \text{Sijoitetaan $y=\frac{13}{7}$.} \\
-x+\frac{65}{7}&= 2 && \\
-x&= -\frac{51}{7}&& \\
x&= \frac{51}{7}&&
\end{align*}
\end{esimratk}
\begin{esimvast}
Yhtälön ratkaisu on $x= 51/7$, $y=13/7$.
\end{esimvast}
\end{esimerkki}

\begin{esimerkki}
Määritä suorien $-2x-y= 4$ ja $3x-2y=1$ leikkauspiste.
\begin{esimratk}
Saadaan ratkaistavaksi yhtälöpari
\[
\left\{
\begin{aligned}
-2x-y&= 4 \\
3x-2y&= 1.
\end{aligned}
\right.
\]
Käytetään \termi{yhteenlaskumenetelmä}{yhteenlaskumenetelmää}, jolla voidaan eliminoida toinen tuntemattomista.
\begin{align*}
  &\left\{
    \begin{aligned}
    2x-y&= 4 && \ppalkki \cdot 3\\
    3x-2y&= 1 && \ppalkki \cdot (-2)
    \end{aligned}
  \right. \\
  &\left\{
    \begin{aligned}
    6x-3y&= 12 && \ppalkki \text{Lasketaan yhtälöt yhteen.}\\
    -6x+4y&= -2.&&
    \end{aligned}
  \right.\\
  &y= 10 \\
\end{align*}
Sijoitetaan $y=10$ jompaan kumpaan alkuperäisistä yhtälöistä, jotta saadaan ratkaistua $x$. Käytetään vaikkapa yhtälöä $3x-2y=1$:
\begin{align*}
3x-2y&=1 && \ppalkki \text{Sijoitetaan $x=10$.} \\
3x-20&=1 && \\
3x&=21 && \\
x&=7 &&.
\end{align*}
\end{esimratk}
\begin{esimvast}
Yhtälön ratkaisu on $x=7$, $y=10$.
\end{esimvast}
\end{esimerkki}

%EDELLISEN ESIMERKIN ULKOASUA PITÄÄ VIELÄ MUOKATA. TASAUKSET EIVÄT TOIMI JA LISÄKSI YHTÄLÖPARIN ALLE PITÄISI SUMMAUKSEN KOHDALLA SAADA VIIVA. SAMA ONGELMA MYÖHEMMISSÄ ESIMERKEISSÄ.

\subsection*{Yhtälöparin ratkaisujen määrä ja geometrinen tulkinta}

Aiemmin on todettu, että normaalimuotoinen ensimmäisen asteen yhtälö voidaan tulkita suorana
$xy$-tasossa. Näin ollen lineaariselle yhtälöparille on geometrinen tulkinta: sen
ratkaisut ovat ne tason pisteet, joissa yhtälöitä vastaavat
suorat leikkaavat. Näitä voi olla
$0$ (suorat ovat yhdensuuntaiset, mutta eivät sama suora),
$1$ (suorat eivät ole yhdensuuntaiset) tai
äärettömän monta (suorat ovat sama suora).

\laatikko{Yhtälöparilla on 0, 1 tai äärettömän monta ratkaisua.}

Tarkastellaan esimerkkiä yhtälöryhmästä, jolla ei ole lainkaan ratkaisuja.

\begin{esimerkki}
Ratkaise yhtälöpari
\[
\left\{
\begin{aligned}
x-2y&= 1 \\
5x-10y&= 2.
\end{aligned}
\right.
\]
\begin{esimratk}
Käytetään yhteenlaskumenetelmää:
\begin{align*}
&\left\{
\begin{aligned}
-x+2y&= 1 && \ppalkki \cdot 5 \\
5x-10y&= 2. &&
\end{aligned}
\right. \\
&\left\{
\begin{aligned}
-5x-10y&= 5 && \ppalkki \cdot \text{Lasketaan yhtälöt yhteen.} \\
5x-10y&= 2. &&
\end{aligned}
\right. \\
&0=7 \\
\end{align*}
Koska päädytään mahdottomaan yhtälöön, yhtälöparilla ei ole ratkaisua.
\end{esimratk}
\begin{esimvast}
Yhtälöparilla ei ole ratkaisua.
\end{esimvast}
\end{esimerkki}

Edellisessä esimerkissä yhtälöryhmällä ei ollut lainkaan ratkaisuja. Geometrisesti tämä tarkoittaa sitä, että suorilla $x-2y= 1$ ja $5x-10y= 2$ ei ole leikkauspisteitä. Ne ovat siis yhdensuuntaiset.
Tämä voi nähdä myös kirjoittamalla suorien yhtälöt muodossa
\begin{align*}
  x-2y &= 1 \\
  2y &= x - 1 \\
  y &= \frac{1}{2}x -1
\end{align*}
ja
\begin{align*}
  5x-10y &= 2 \\
  10y &= 5x -2 \\
  y &= \frac{1}{2}x - \frac{1}{5},
\end{align*}
josta nähdään että suorien kulmakertoimet ovat samat.

TÄHÄN KUVA?

Tutkitaan vielä yhtälöparia, jolla on äärettömän monta ratkaisua.

\begin{esimerkki}
Ratkaise yhtälöpari
\[
\left\{
\begin{aligned}
-3x+y&= -1 \\
6x-2y&= 2.
\end{aligned}
\right.
\]
\begin{esimratk}
Käytetään sijoitusmenetelmää. Ensimmäisestä yhtälöstä saadaan $y=3x-1$. Sijoitetaan tämä toiseen yhtälöön:
\begin{align*}
6x-2y&= 2 && \ppalkki \text{sijoitetaan $y=3x-1$} \\
6x-2(3x-1)&= 2 && \\
6x-6x+2&= 2 && \\
2&= 2. &&
\end{align*}
\end{esimratk}
Näin saadusta yhtälöstä $2=2$ ei voikaan ratkaista tuntematonta $x$ niin kuin oli tarkoitus. Yhtälö ei anna mitään lisätietoa tuntemattomista. Tämä johtuu siitä, että alkuperäiset yhtälöt $-3x+y= -1$ ja $6x-2y= 2$ ovat yhtäpitäviä. Ensimmäisestä saadaan jälkimmäinen kertomalla luvulla $-2$.

Oleellisesti tarkasteltavana onkin vain yksi yhtälö, $-3x+y= -1$. Yhtälöparin ratkaisuja ovat kaikki ne luvut $x$ ja $y$, jotka toteuttavat tämän yhtälön. Ratkaisuja ovat esimerkiksi $x=0$, $y=-1$ ja $x=1$, $y=2$. Ratkaisuja on äärettömän monta.
\begin{esimvast}
Ratkaisuja on äärettömän monta.
\end{esimvast}
\end{esimerkki}

Edellisessä esimerkissä yhtälöparilla oli äärettömän monta ratkaisua. Geometrisesti tämä tarkoittaa sitä, että yhtälöt $-3x+y= -1$ ja $6x-2y= 2$ ovat saman suoran yhtälö. Siten kaikki suoralla $-3x+y= -1$ (tai yhtä hyvin suoralla $6x-2y= 2$) olevat pisteet toteuttavat yhtälöparin.



\subsection*{Useamman kuin kahden yhtälön yhtälöryhmät}

Kun ratkaistavia yhtälöitä voi olla useampia kuin kaksi, puhutaan yleisesti \termi{yhtälöryhmä}{yhtälöryhmästä}. Yhtälöpari on yhtälöryhmän erikoistapaus. \footnote{Joskus myös yhtälöparia kutsutaan yhtälöryhmäksi, kun ei haluta korostaa yhtlöiden lukumäärää.}

Tässä tarkastellaan lähinnä kolmen yhtälön lineaarisia yhtälöryhmiä. Tätä useamman yhtälön
lineaarisista yhtälöryhmistä esitetään joitakin helppoja esimerkkejä. 

Tähän asti esimerkeissä on käsitelty yhtälöryhmiä, joissa muuttujia on ollut yhtä monta kuin yhtälöitä, mutta yhtälöitä ja muuttujia voi olla myös eri määrä, mikä vaikuttaa yhtälöryhmän mahdollisten ratkaisujen lukumäärään.

Yleisesti ottaen yhtälöryhmiä ei käytännössä ratkaista tällä kurssilla esitetyin keinoin. Useimmiten yhtälöryhmiä ratkaistaan likimääräisesti tietokoneella käyttäen numeerista matriisilaskentaa.
%LISÄMATERIAALEIHIN MAININTA, ESITTELY? Soveltava kurssi, Vapaa Matikka MaaN.N.: Lukion lineaarialgebraa?


\begin{esimerkki}
Ratkaise lineaarinen yhtälöryhmä
\[
\left\{
\begin{aligned}
4x+2y+5z&=0 \\
6x-3y-z&= 23 \\
x-2y+3z&= -1.
\end{aligned}
\right.
\]
\begin{esimratk}
Käytetään yhteenlaskumenetelmää kahteen ylimpään yhtälöön ja eliminoidaan niistä tuntematon $x$:
\begin{align*}
&\left\{
\begin{aligned}
4x+2y+5z&=0 && \ppalkki \cdot 3 \\
6x-3y-z&= 23 && \ppalkki \cdot (-2) \\
\end{aligned}
\right. \\
&\left\{
\begin{aligned}
12x+6y+15z&=0 \\
-12x+6y+2z&= -46 \\
\end{aligned}
\right. \\
&12y+17z=-46 \\
\end{align*}

Tehdään sitten sama toiselle ja kolmannelle yhtälölle:
\begin{align*}
&\left\{
\begin{aligned}
6x-3y-z&= 23 && \\
x-2y+3z&= -1 && \ppalkki \cdot (-6)
\end{aligned}
\right. \\
&\left\{
\begin{aligned}
6x-3y-z&= 23 \\
-6x+12y-18z&=6
\end{aligned}
\right. \\
&9y-19z=29. \\
\end{align*}

Nyt on saatu kaksi yhtälöä, joissa ei ole tuntematonta $x$. Ratkaistaan näin muodostuva yhtälöpari:
\begin{align*}
&\left\{
\begin{aligned}
12y+17z&=-46 && \ppalkki \cdot 3 \\
9y-19z&=29 && \ppalkki \cdot (-4) \\
\end{aligned}
\right. \\
&\left\{
\begin{aligned}
36y+51z&=-138  \\
-36y+76z&=-116  \\
\end{aligned}
\right. \\
&127z=-254 \\
&z=-2 \\
\end{align*}

Nyt tiedetään, että $z=-2$. Sijoitetaan tämä aiemmin saatuun kahden tuntemattoman yhtälöön $12y+17z=-46$:
\begin{align*}
12y+17z&=-46 && \ppalkki \text{Sijoitetaan $z=-2$.} \\
12y-34&=-46 && \\
12y&=-12 && \\
y&=-1 && \\
\end{align*}

Lopuksi sijoitetaan $y=-1$ ja $z=-2$ johonkin alkuperäisistä yhtälöistä, vaikkapa ensimmäiseen yhtälöön:
\begin{align*}
4x+2y+5z&=0 && \ppalkki \text{Sijoitetaan $y=-1$ ja $z=-2$.} \\
4x-2-10&=0 && \\
4x&=12 && \\
x&=3 && \\
\end{align*}

Nyt on siis saatu ratkaisu $x=3$, $y=-1$, $z=-2$. Tarkistetaan vielä, että nämä luvut tosiaankin toteuttavat kaikki alkuperäisen yhtälöryhmän
\[
\left\{
\begin{aligned}
4x+2y+5z&=0 \\
6x-3y-z&= 23 \\
x-2y+3z&= -1.
\end{aligned}
\right.
\]
yhtälöt. Tämä tehdään sijoittamalla luvut yhtälöiden vasemmalle puolella ja tarkistamalla, että tulos täsmää yhtälön oikean puolen kanssa.

Koska $4 \cdot 3+2\cdot(-1)+5\cdot(-2)=12-2-10=0$, ne toteuttavat ensimmäisen yhtälöistä. Samalla tavalla $6 \cdot 3-3\cdot(-1)-\cdot(-2)=18+3+2=23$, joten luvut toteuttavat myös toisen yhtälön. Lopuksi todetaan, että $3-2\cdot(-1)+3\cdot(-2)=3+2-6=0-1$, ja siten kolmaskin yhtälöistä toteutuu.

\end{esimratk}
\begin{esimvast}
Yhtälöryhmän ratkaisu on $x=3$, $y=-1$, $z=-2$.
\end{esimvast}
\end{esimerkki}

\begin{tehtavasivu}

\subsubsection*{Opi perusteet}

\begin{tehtava}
  Mitkä seuraavista ovat lineaarisia yhtälöitä?
  \begin{alakohdat}
    \alakohta{$2x^2 + y +1 = 0$}
    \alakohta{$y = 0$}
    \alakohta{$3x - 2y = 5$}
  \end{alakohdat}
  \begin{vastaus}
    \begin{alakohdat}
      \alakohta{Ei.}
      \alakohta{On.}
      \alakohta{On.}
    \end{alakohdat}
  \end{vastaus}
\end{tehtava}

\begin{tehtava}
    Ratkaise yhtälöparit.
    \begin{align*}
        x-2y &= 0 \\
        -2x+y+3 &=0
    \end{align*}
    \begin{vastaus}
        $x = 2, \, y = 1$
    \end{vastaus}
\end{tehtava}

\begin{tehtava}
    Ratkaise yhtälöpari.
    \begin{align*}
        x+y+1 &= 0 \\
        x+2y+1 &=0
    \end{align*}
    \begin{vastaus}
        $x = -1, \, y = 0$
    \end{vastaus}
\end{tehtava}

\begin{tehtava}
    Ratkaise yhtälöpari.
    \begin{align*}
        2x+5y+1 &= 0 \\
        2x+2y+7 &=0
    \end{align*}
    \begin{vastaus}
        $x = -\frac{11}{2}, \, y = 2$
    \end{vastaus}
\end{tehtava}

\begin{tehtava}
    Ratkaise yhtälöpari.
    \begin{align*}
        2a+3b &= 8 \\
        6a+2b &= -4
    \end{align*}
    \begin{vastaus}
        $a = -2, \, b = 4$
    \end{vastaus}
\end{tehtava}

\begin{tehtava}
    Ratkaise yhtälöpari.
    \begin{align*}
        \frac{x}{3}+\frac{y}{7} + 1 &= 3 \\
        x - \frac{y-1}{3} &= -5
    \end{align*}
    \begin{vastaus}
        $x = -45, \, y = 119$
    \end{vastaus}
\end{tehtava}

\begin{tehtava}
	Ratkaise yhtälöpari.
	\begin{align*}
		x^2-y+1 &= 0 \\
		x+y-2 &= 0
	\end{align*}
	\begin{vastaus}
		$x = \frac{-1+\sqrt{5}}{2}, y = \frac{10-2\sqrt{5}}{4}$ tai $x = \frac{-1-\sqrt{5}}{2}, y = \frac{10+2\sqrt{5}}{4}$
	\end{vastaus}
\end{tehtava}

\begin{tehtava}
    Ratkaise yhtälöpari. $t \in \rr$ on vapaa parametri, joka saa sisältyä vastaukseen.
    \begin{align*}
        x+2y-t-1 &= 0 \\
        x+y+t^2 &=0
    \end{align*}
    \begin{vastaus}
        $x = -2t^2-t-1, \, y = t^2+t+1$
    \end{vastaus}
\end{tehtava}

\begin{tehtava}
    Ratkaise yhtälöryhmä.   
    \begin{align*}
        x+2y+1 &=0 \\
        x+2z+3 &=0 \\
        y+2z+5 &=0
    \end{align*}
    \begin{vastaus}
        $x = 1, \, y = -1, \, z = -2$
    \end{vastaus}
\end{tehtava}

\subsubsection*{Hallitse kokonaisuus}

\begin{tehtava}
    Ratkaise yhtälöryhmä.
    \begin{align*}
        x+y+z+8 &= 0 \\
        x+y+6 &=0 \\
        x+z-70 &=0
    \end{align*}
    \begin{vastaus}
        $x = 72, \, y = -78, \, z = -2$
    \end{vastaus}
\end{tehtava}

\begin{tehtava}
    Ratkaise yhtälöryhmä.
    \begin{align*}
        x+y+2z+12 &= 0 \\
        2x+2y+3z+1 &=0 \\
        3x-4 &=0
    \end{align*}
    \begin{vastaus}
        $x = \frac{4}{3}, \, y = \frac{98}{3}, \, z = -23$
    \end{vastaus}
\end{tehtava}

\begin{tehtava}
    Ratkaise yhtälöryhmä.
    \begin{align*}
        2x+3y+5z+8 &= 0 \\
        3x+5y+8z &=0 \\
        x+y-1 &=0
    \end{align*}
    \begin{vastaus}
        $x = -\frac{63}{2}, \, y = \frac{65}{2}, \, z = -\frac{17}{2}$
    \end{vastaus}
\end{tehtava}

\begin{tehtava}
	Ratkaise yhtälöryhmä.
	\begin{align*}
		x+w+3 &= 0 \\
		x+y+z &= 0 \\
		y-w-3 &= 0 \\
		w-2z+5 &= 0
	\end{align*}
	\begin{vastaus}
		$x=2, y=-2, z=0, w=-5$
	\end{vastaus}
\end{tehtava}


\begin{tehtava}
	Ratkaise yhtälöryhmä.
	\begin{align*}
		x+2y-z &= 0 \\
		x+3y-w &= 0 \\
		x+y+z+w-4 & = 0 \\
		x+2y+2z+2w-6 &= 0
	\end{align*}
	\begin{vastaus}
		$x=2, y=-\frac13, z=\frac43, w=1$
	\end{vastaus}
\end{tehtava}

\begin{tehtava}
	Ratkaise yhtälöryhmä.
    	\begin{align*}
        	x+y+tz &=1 \\
        	x+ty+z &=1 \\
        	tx+y+z &=1
    	\end{align*}
	\begin{vastaus}
		$x = y = z = \frac{1}{t+2}$, kun $1 \neq t \neq -2$. Kun $t = -2$ yhtälöryhmällä ei ole ratkaisuja.
		Kun $t = 1$ kaikki kolmikot muotoa $x = r$, $y = s$ ja $z = 1-r-s$, jollain reaaliluvuilla $r$ ja $s$ ovat ratkaisuja.
	\end{vastaus}
\end{tehtava}

\begin{tehtava}
  Yhden muuttujan lineaarinen yhtälö $ax +b = 0$ määrittää pisteen ($x = \frac{-b}{a}$) lukusuoralla.
  Kahden muuttujan lineaarinen yhtälö puolestaan $ax +by +c = 0$ määrittää kaksiulotteisessa tasossa suoran eli pistejoukon $(x,y) = (x, \frac{-a}{b}x - \frac{c}{b})$.
  Kolmen muuttujan avulla voidaan määrittää kolmiulotteisen \emph{avaruuden} piste $(x, y, z)$.
  Tutki, minkä pistejoukon kolmen muuttujan lineaarinen yhtälö määrittää, toisin sanoen, mitkä kolmiulotteisen avaruuden pisteet $(x, y, z)$ toteuttavat yhtälön $ax +by +cz +d = 0$
  \begin{alakohdat}
    \alakohta{kun $a = 1, b = 0, c= 0, d=0$?}
    \alakohta{kun $a = 1, b = 1, c= 0, d=1$?}
    \alakohta{kun $a, b, c$ ja $d$ ovat kaikki erisuuria kuin nolla?}
  \end{alakohdat}
  \begin{vastaus}
    \begin{alakohdat}
      \alakohta{Saadaan yhtälö  $x = 0$ ja siten pistejoukko $(0, y, z)$ ($y$ ja $z$-koordinaattia ei rajoiteta mitenkään, ne voivat olla mitä tahansa reaalilukuja), joka on kolmiulotteisessa $xyz$-avaruudessa $y$- ja $z$-akselien suuntainen taso. }
      \alakohta{$x + y +1 = 0$ ja $ (x, -x -1, z)$. Yhtälö määrää $xy$-tasossa suoran $y = -x-1$, ja $z$ voi olla mikä tahansa reaaliluku. Saadaan siis $z$-akselin suuntainen taso, joka kulkee $xy$-tason suoran $y = -x-1$ kautta.}
      \alakohta{Kukin muuttuja voidaan kirjoittaa kahden muun avulla, esim. $ z= -\frac{(ax + by + d)}{c}$, jolloin saadaan pistejoukko $(x, y, -\frac{(ax + by + d)}{c})$. Siis kun valitaan mitkä tahansa $x$- ja $y$-koordinaatit, $z$-koordinaatti on määrätty. Yhtälö määrittää edelleen tason.}
    \end{alakohdat}
  \end{vastaus}
\end{tehtava}

\subsubsection*{Sekalaisia tehtäviä}

TÄHÄN TEHTÄVIÄ SIJOITTAMISTA ODOTTAMAAN

\end{tehtavasivu}
