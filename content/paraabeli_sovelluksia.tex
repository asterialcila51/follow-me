\section{Paraabelin sovelluksia}

\laatikko{
KIRJOITA TÄHÄN LUKUUN

\luettelo
§ TEHTY paraabelin huippu on kohdassa $x=-b/2a$, todistus
§ (Jo edellisessä kappaleessa? Vai pitäisikö siirtää tähän?) paraabelin yhtälön huippumuoto $y-y_0=a(x-x_0)^2$
§ TEHTY paraabelin yhtälön ratkaiseminen kolmen pisteen avulla
§ soveltavia tehtäviä, ne iänikuiset holvikaaret jne.
\loppu
KIITOS!}

\subsection{Paraabelin huipun sijainti}
Joissain tehtävissä hyödylliseksi osoittautuu tulos, että paraabelin $y = ax^2 +b +c$ huipun $(x_0, y_0)$ $x$-koordinaatti on aina
\[x_0 = - \frac{b}{2a}\]
ja $y$-koordinaatti
\[y_0 = c - \frac{b^2}{4a}.\]

%%% FIXME MAA2:n liitteistä toteamus että x= \frac{-b}{2a} on neliön minimi tjsp tähän?

Huipun sijainnin kohdassa $x = - \frac{b}{2a}$ voi osoittaa tutkimalla paraabelin määräävän polynomifunktion ääriarvoja funktion derivaatan avulla; derivaattaan tutustutaan kurssilla MAA7. Saman tuloksen näkee myös paraabelin yhtälön huippumuodon $y - y_0 = a(x- x_0)^2$ ja normaalimuodon $y = ax^2 +bx +c$ yhtäpitävyydestä. Palautetaan edellisestä kappaleesta mieleen, että huippumuodon saattaa kirjoittaa auki

\begin{align*}
y-y_0 &= a(x-x_0)^2 \\
y-y_0 &= a(x^2 - 2x_0x + x_0^2) \\
y &= ax^2 - 2ax_0 x + ax_0^2 +y_0.
\end{align*}
Nyt
\[ax^2 - 2ax_0 x + ax_0^2 +y_0 = ax^2 +bx +c\]
kun $b = -2ax_0$ ja $c = ax_0^2 +y_0$, ja siis
\[x_0 = -\frac{b}{2a}\]
ja sijoittamalla tämä saadaan myös $y_0$
\begin{align*}
y_0 &= c - ax_0^2 = c -a\left(-\frac{b}{2a}\right)^2 \\
&= c - \frac{ab^2}{4a^2} \\
&= c - \frac{b^2}{4a}.
\end{align*}

\subsection{Paraabelin yhtälön ratkaiseminen kolmen pisteen avulla}

\begin{esimerkki}
	Paraabeli kulkee pisteiden $(0,1)$, $(2,0)$ ja $(4,1)$ kautta. Muodosta paraabelin yhtälö.
	
	\begin{esimratk} 
		Jokainen annettu piste toteuttaa saman paraabelin yhtälön $y - y_0 = k(x-x_0)^2$. Näin ollen kannattaa sijoittaa kunkin pisteen koordinaatit paraabelin yhtälöön ja muodostaa kolmen pisteen muodostamista yhtälöistä yhtälöryhmä:
		\[
		\left\{
		\begin{aligned}
		1 - y_0 = k(0 - x_0)^2 \\
		0 - y_0 = k(2 - x_0)^2 \\
		1 - y_0 = k(4 - x_0)^2
		\end{aligned}
		\right.
		\]

		Yhtälöryhmä ratkaistaan samalla lailla kuten aiemminkin:
		
		Esimerkiksi voidaan ratkaista $y_0$ ensimmäisestä yhtälöstä ja sijoittaa se kolmanteen.

\begin{align*}
	1 - y_0 &= k(0 - x_0)^2 &&\ppalkki \text{1. yhtälö}\\
	y_0 &= 1- kx^2_0	 \\
    \intertext{Sijoitetaan tämä kolmanteen:}
	1-y_0 &= k(4-x_0)^2 &&\ppalkki \text{Avataan binomin neliö} \\
	1-y_0 &= k(16 - 8x_0 + x_0^2) &&\ppalkki \text{ja sijoitetaan }y_0 \\
	1-1+kx^2_0 &= 16k - 8kx_0 + x^2_0 &&\ppalkki -kx^2_0 \\
	-8kx_0 + 16k &= 0 &&\ppalkki \cdot \, \frac{1}{8k}, \, k \neq 0 \\
	x_0 &= 2 \\
	y_0 &= 1- kx^2_0 = 1 - 4k	
\end{align*}	

		 Sijoitetaan saadut $x_0$:n ja $y_0$:n arvot keskimmäiseen yhtälöön.

\begin{align*}	
	0 - y_0 &= k(2 - x_0)^2 \\
	0 - (1 - 4k) &= k(2-2)^2 \\
	4k -1 &= 0 \\
	k &= \frac{1}{4} \\
	y_0 &= -k(2 - x_0)^2 = - \frac{1}{4} \cdot 0 = 0
\end{align*}

		Sijoitetaan lopulliset arvot paraabelin yhtälöön:

\begin{align*}	
    y - y_0 &= k(x-x_0)^2 \\
	y - 0 &= \frac{1}{4}(x-2)^2 \\
	y &= \frac{1}{4}(x^2-4x+4) \\
	y &= \frac{1}{4}x^2 - x + 1
\end{align*}


		\begin{esimvast}
			$y=\frac{1}{4}x^2 - x + 1$
			
			Aivan yhtä hyvin voitaisiin käyttää myös yhtälöä $ax^2 + bx +c$.
		\end{esimvast}


	\end{esimratk}
\end{esimerkki}




\begin{tehtavasivu}

\subsubsection*{Opi perusteet}

\begin{tehtava}
Arkkitehti Guggenheim suunnittelee uuteen taidemuseoon kahta paraabelin muotoista holvikaarta. Holvikaarten leveys on 5 metriä ja korkeus 7 metriä, ne ovat puolen metrin päässä toisistaan ja ne ovat sijoitettu symmetrisesti (y-akseliin nähden) julkisivulle. Määritä holvikaarten yhtälöt.
\begin{vastaus}
%holvikaaret puoli metriä toisistaan, siis etäisyys y-akselista 0,25m.
%Tässä ensimmäinen piste, toinen piste on 5,25m päässä, ja kolmas on (2,75; 7). %Riittää määrittää vain yksi paraabeli, ja toisen saa x_0:n vastaluvusta.
$y-7 = -\frac{6,25}{7}(x - 2,75)^2$ ja $y-7 = -\frac{6,25}{7}(x + 2,75)^2$
\end{vastaus}
\end{tehtava}



\subsubsection*{Hallitse kokonaisuus}

\subsubsection*{Sekalaisia tehtäviä}

TÄHÄN TEHTÄVIÄ SIJOITTAMISTA ODOTTAMAAN

%%% Hei, ei näiden todistustehtävien tarvitse olla vaikeita! Testataan onko luettu myös ymmärretty. :D
\begin{tehtava}
Osoita, että jos toisen asteen yhtälöllä on kaksi nollakohtaa, sen kuvaajan paraabelin huipun $x$-koordinaatti on nollakohtien puolivälissä.
\begin{vastaus}
    \emph{Vihje:} Toisen asteen yhtälön ratkaisukaava.
\end{vastaus}
\end{tehtava}

%%% Jatkoa edellisen teemaan. Tämäkin onnistuu varmaan kuinka monella tavalla.
\begin{tehtava}
Osoita miksi paraabeli $y = ax^2 +bx +c$ on symmetrinen huippunsa kautta kulkevan $y$-akselin suuntaisen suoran suhteen.
\begin{vastaus}
    \emph{Vihje:} Edelleen, toisen asteen yhtälön ratkaisukaava.
\end{vastaus}
\end{tehtava}

%%% Helppoa intuitiota funktion ääriarvopisteiden etsimiseen
\begin{tehtava}
Usein huomataan, että halutaan tietää missä pisteessä jokin funktio saa pienimmän tai suurimman arvonsa.

Onko seuraavilla funktioilla suurinta tai pienintä arvoa? Jos on, mikä se on, ja millä $x$:n arvolla se saavutetaan?
\alakohdat
    § $f(x) = 4x^2 - 8x + 8$
    § $f(x) = -x^2 - 3x - 1$
    § $f(x) = (x-2)(x-3)$
\loppu

Kurssilla MAA??? (derivaattakurssi?) tutustutaan tarkemmin funktion ääriarvojen (pieninten ja suurinten arvojen) etsimiseen, erityisesti myös silloin kun funktio ei ole toisen asteen polynomi.
	\begin{vastaus}
	\alakohdat
	    § Pienin arvo $4$ pisteessä $x = 1$ (funktion kuvaaja on ylöspäin aukeava paraabeli).
	    § Suurin arvo $\frac{5}{4}$ pisteessä $\frac{-3}{2}$.
	    § Pienin arvo $\frac{-1}{4}$ pisteessä $\frac{5}{2}$.
	\loppu
	\end{vastaus}
\end{tehtava}

%%%Jatkona edelliseen, tähän pari soveltavampaa 'geometrista' ääriarvotehtävää jossa funktio toista astetta

%%%Ei tämäkään välttämättä vaikea:
\begin{tehtava}
Käytettävissä on 100 tulitikkua, joista jokainen on 4~cm pitkä, ja niistä halutaan muodostaa pinta-alaltaan mahdollisimman suuria kuvioita. Tikkuja saa katkaista mistä kohtaa tahansa.
\alakohdat
    § Kuinka suuren alan peittää tikuista rajattu mahdollisimman suuri suorakulmio, ja mitkä ovat sen sivujen pituudet?
    § Muuten sama kuin edellä, paitsi että tikuista rajataan kaksi samankokoista suorakulmioita joilla on yksi yhteinen tikuista tehty sivu, ja halutaan suorakulmioiden yhteispinta-ala mahdollisimman suureksi?
    § Sama kuin (b), paitsi että suorakulmioiden ei tarvitse olla samankokoisia.
\loppu
	\begin{vastaus}
	\alakohdat
	    %%%Saattaa huomata ilman laskemistakin että suurin ala on neliöllä. 
	    %%%Lisäksi huomattava ettei kysytty alaa tikkuina!
        § Neliö, jonka sivun pituus 100~cm ja pinta-ala 1~m$^2$
        %%%Ei kovin vaikea edellisen sovellus, kunhan huomaa millaista kuviota tarkoitetaan.
        § Kummankin lyhyempi sivu 50~cm ja pitempi sivu $66\frac{2}{3}$~cm$^2$. Yhteispinta-ala $\frac{2}{3}$~m$^2$.
        %%%Kompa. Sillä missä kohtaa sijaitsee suuren suorakulmion kahdeksi jakava sivu ei ole merkitystä pinta-alan tai sivujen yhteispituuksien kannalta!
        § Sama kuin edellisessä.
    \loppu
	\end{vastaus}
\end{tehtava}

%%%Hieman vaikeampi muunnos edellisestä olisi jos tikkujen katkaisemista rajoitetaan...?

%%%Sitten lukion fysiikkaan liittyviä tehtäviä:
%%%*Putoavan kappaleen sijainti
%%%*Tykinkuulan/pallon/tjsp heitto/lento(/...)rata

\end{tehtavasivu}