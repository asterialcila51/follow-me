\section{Todistuksia}

\subsection*{Pisteen etäisyys suorasta analyyttisesti}

Lasketaan pisteen $P = (x_0, y_0)$ etäisyys suorasta $l$: $ax+by+c=0$.

Olkoon $Q$ suoralla $l$ siten, että $l$ ja $PQ$ ovat kohtisuorassa. Huomataan, että jos piste $R$ on suoralla $l$, Pythagoraan lauseen mukaan
\[
PR^2 = PQ^2+QR^2 \geq PQ^2,
\]
jolloin myös $PR \geq PQ$. Siis $PQ$ on määritelmän nojalla pisteen $P$ etäisyys suorasta $l$.

Suorat $PQ$ ja $l$ ovat kohtisuorassa. Jos $l$ ei ole $x$-, eikä $y$-akselin suuntainen, eli $a,b \neq 0$ sen kulmakerroin on $-\frac{a}{b}$. Koska suoran ja normaalin kulmakerrointen tulo on $-1$, suoran $PQ$ kulmakerroin on
\[
-\frac{1}{\frac{-a}{b}} = \frac{b}{a}.
\]
Se kulkee lisäksi pisteen $(x_0,y_0)$ kautta, joten sen yhtälö on
\[
y-y_0 = \frac{b}{a}(x-x_0)
\]
tai normaalimuodossa
\[
bx-ay+ay_0-bx_0 = 0.
\]
Toisaalta, jos $a = 0$, suora $PQ$ on muotoa $x+C$, jollain reaaliluvulla $C$; koska se lisäksi kulkee pisteen $P$ kautta, sen on oltava edellistä muotoa. Sama päättely voidaan toistaa kun $b = 0$, jolloin nähdään, että kaava normaalille pätee myös jos $a = 0$ tai $b = 0$.

Koska $Q$ kuuluu suoralle $l$ ja sen normaalille, sen koordinaattien on toteutettava yhtälöpari
\[
\left\{    
    \begin{array}{rcl}
        ax_q + by_q + c &=&0 \\
        bx_q-ay_q+ay_0-bx_0 &=& 0 \\
    \end{array}
    \right.
\]

Yhtälöpari voidaan ratkaista yhteenlaskumenetelmällä kertomalla ylempi yhtälö $a$:lla ja alempi $b$:llä.
\[
\left\{    
    \begin{array}{rcl}
        a^2x_q + aby_q + ac &=&0 \\
        b^2x_q-aby_q+aby_0-b^2x_0 &=& 0 \\
    \end{array}
    \right.
\]
ja laskemalla yhtälöt puolittain yhteen
\begin{align*}
 a^2x_q + aby_q + ac + b^2x_q-aby_q+aby_0-b^2x_0 &= 0 \\
 (a^2+b^2)x_q &= -ac-aby_0+b^2x_0 \\
 x_q = \frac{-ac-aby_0+b^2x_0}{(a^2+b^2)}.
\end{align*}

Vastaavasti
\[
y_q = \frac{-bc+a^2y_0-abx_0}{(a^2+b^2)}.
\]
Nyt
\begin{align*}
  PQ &= \sqrt{(x_q-x_0)^2+(y_q-y_0)^2} \\
  &= \sqrt{\Big(\frac{-ac-aby_0+b^2x_0-(a^2+b^2)x_0}{a^2+b^2}\Big)^2+\Big(\frac{-bc+a^2y_0-abx_0-(a^2+b^2)y_0}{a^2+b^2}\Big)^2}, \\
  \intertext{jota voidaan sieventää:}
  & = \sqrt{\frac{(-ac-aby_0-a^2x_0)^2+(-bc-abx_0-b^2 y_0)^2}{\left(a^2+b^2\right)^2}} \\
  & = \frac{\sqrt{(-ac-aby_0-a^2x_0)^2+(-bc-abx_0-b^2 y_0)^2}}{a^2+b^2} \\
  \intertext{ja huomataan että sopivasti ryhmittelemällä saadaan} 
  & = \frac{\sqrt{(-a\textcolor{blue}{c}-a\textcolor{blue}{by_0}-a\textcolor{blue}{ax_0})^2+
  (-b\textcolor{blue}{c}-b\textcolor{blue}{ax_0}-b\textcolor{blue}{by_0})^2}}{a^2+b^2}. \\
  & = \frac{\sqrt{(a(-c-by_0-ax_0))^2+(b(-c-ax_0-by_0))^2}}{a^2+b^2} \\
  & = \frac{\sqrt{a^2(-c-by_0-ax_0)+b^2(-c-ax_0-by_0)}}{a^2+b^2}, \\
  \intertext{jolloin  $\sqrt{a^2 + b^2}$ supistuu pois}
  & = \frac{\sqrt{(a^2+b^2)(-c-by_0-ax_0)^2}}{a^2+b^2} \\
  & = \frac{\sqrt{(a^2+b^2)}}{a^2+b^2}\sqrt{(-c-by_0-ax_0)^2} \\
  & = \frac{|ax_0+by_0+c|}{\sqrt{a^2+b^2}}.
\end{align*}
