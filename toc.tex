\providecommand{\lukufilter}[2]{#2} % ylikirjoitetaan kaanna_luku.sh -skriptistä.
\newcommand{\luku}[1]{\lukufilter{#1}{\input{content/#1}}} % luku

\Opensolutionfile{ans}[content/LIITE_vastaukset]

\chapter{Esitietoja}
	\luku{itseisarvo}
	\newpage \luku{epayhtalo}
	% itseisarvoepäyhtälöt
	\newpage \luku{koordinaatisto}
	% yleistä käyristä, esim. Kartesiuksen lehdestä jotain
	% kahden pisteen välinen etäisyys (Pythagoraan lauseella)

\chapter{Suorat ja lineaariset yhtälöryhmät}
	\luku{suora}
	% ratkaistu muoto y = kx + b, kulmakerroin ja vakiotermi
	% nollakohdat ja leikkauspisteet
	% vaaka- ja pystysuorat
	\newpage \luku{suora_esitykset}
	% esitys y-y_0=k(x-x_0)
	% esitys ax + by + c = 0 (normaalimuoto)
	\newpage \luku{pisteen_etaisyys}
	% kaava pisteen etäisyydelle suorasta
	\newpage \luku{suora_asema}
	% suorien keskinäinen asema, yhdensuuntaiset suorat
	% suoralle ja sen normaalille k_1 * k_2 = -1
	\newpage \luku{yhtaloryhma}
	% sijoitusmenetelmä
	% yhtälöiden laskeminen yhteen
	% ratkaisujen määrä

\chapter{Toisen asteen käyrät}
	\luku{diskriminantti}
	% diskriminantin pikakertaus
	\newpage \luku{ympyra}
	% ympyrän yhtälö määritelmästä
	% ensin origokeskeinen
	% keskipiste ja säde muissa tapauksissa neliöksi täydentämällä
	\newpage \luku{ympyra_suora}
	% suoran ja ympyrän leikkauspisteet
	% tangentit
	\newpage \luku{paraabeli}
	% merkitys toisen asteen polynomin kuvaajana
	% geometrisen määritelmän maininta
	\newpage \luku{paraabeli_sovelluksia}
	% huipun x-koordinaatti on -b/2a
		% todistus liitteeksi -Ville
		% todistus tähän -Niko
	% paraabelin tangentit
	\newpage \luku{paraabeli_kaannetty}
	% paraabeli x = ay^2  +by + c
	\newpage \luku{sekalaista}
	% esimerkkejä ja tehtäviä (erikoisia, vaikeita, yms.)

\Closesolutionfile{ans}

\liitetyyli

\section{Vastaukset} \luku{LIITE_vastaukset}

\newpage \luku{filosofiaa}

\newpage \luku{LIITE_harjoituskokeita}
\newpage \luku{LIITE_ylioppilaskokeita}

% \newpage \luku{filosofiaa} % git add !!!
\newpage \luku{LIITE_tasokayra}
	% esim. piste, kaksi suoraa, tyhjä
\newpage \luku{LIITE_ellipsi}
\newpage \luku{LIITE_hyperbeli}
\newpage \luku{LIITE_kolmioepayhtalo}
\newpage \luku{LIITE_todistuksia}
	% kaava pisteen etäisyydelle suorasta
	% kaikkien paraabelien yhdenmuotoisuus
